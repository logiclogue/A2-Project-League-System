\newpage
\subsection{PHP}
\subsubsection{Models}
\textbf{FixturesGet.php}

Path: /models/FixturesGet.php
{\scriptsize
\begin{lstlisting}
<?php

require_once(dirname(__DIR__) . '/php/Model.php');
require_once(dirname(__DIR__) . '/php/Database.php');
require_once(dirname(__DIR__) . '/superclasses/EloRating.php');

session_start();


/**
 * Model that returns fixture list for a tournament or user.
 * Lists all matches that are still to be played.
 *
 * @class FixturesGet
 * @extends Model
 */
/*&
 * @param player_id {Integer} Find all matches that a user still has to play.
 * @param tournament_id {Integer} Find all matches that still need to be played in a tournament.
 *
 * @return fixtures
 *   @return [].player1_id {Integer} Id of player 1.
 *   @return [].player2_id {Integer} Id of player 2.
 *   @return [].player1_name {String} Full name of player 1.
 *   @return [].player2_name {String} Full name of player 2.
 *   @return [].player1_rating {Integer} Rating of player 1.
 *   @return [].player2_rating {Integer} Rating of player 2.
 *   @return [].expected {Float} Odds of player 1 winning.
 *   @return [].expected_percent {Integer} Percentage chance of player 1 winning.
 *   @return [].tournament_id {Integer} Id of tournament that the match is in.
 *   @return [].tournament_name {String} Name of tournament that the match is in.
 *   @return [].is_league_manager {Boolean} Whether the user is a league manager.
 */
class FixturesGet extends Model
{
	/**
	 * SQL query string that fetches the matches to be played.
	 *
	 * @property query
	 * @private
	 */
	private $query = <<<SQL
		SELECT
		u1.id player1_id,
		u2.id player2_id,
		CONCAT(u1.first_name, ' ', u1.last_name) player1_name,
		CONCAT(u2.first_name, ' ', u2.last_name) player2_name,
		t.id tournament_id,
		t.name tournament_name,
		CASE WHEN tuu.is_league_manager IS NULL THEN FALSE ELSE tuu.is_league_manager END is_league_manager
		FROM users u1

		INNER JOIN users u2
		INNER JOIN tournament_user_maps tu
		ON tu.user_id = u1.id
		INNER JOIN tournament_user_maps tu2
		ON tu2.user_id = u2.id AND tu.tournament_id = tu2.tournament_id
		INNER JOIN tournaments t
		ON t.id = tu.tournament_id
		LEFT JOIN tournament_user_maps tuu
		ON tuu.user_id = :user_id AND t.id = tuu.tournament_id

		WHERE
		tu.is_player = TRUE AND
		tu2.is_player = TRUE AND
		u1.id <> u2.id AND
		CASE WHEN :tournament_id IS NULL THEN TRUE ELSE t.id = :tournament_id END AND
		CASE WHEN :player_id IS NULL THEN u1.id > u2.id ELSE u1.id = :player_id END AND
		(
			SELECT COUNT(*)
			FROM result_user_maps ru1
			INNER JOIN result_user_maps ru2
			ON ru2.result_id = ru1.result_id
			INNER JOIN results r
			ON r.id = ru1.result_id
			WHERE
			ru1.user_id <> ru2.user_id AND
			r.tournament_id = tu.tournament_id AND
			ru1.user_id = u1.id AND
			ru2.user_id = u2.id
		) = 0
SQL;


	/**
	 * Method that gets the user ratings and puts them in every fixture including the expected result for player 1.
	 *
	 * @method attachRatings
	 * @private
	 */
	private function attachRatings() {
		foreach($this->return_data['fixtures'] as &$fixture) {
			$fixture['player1_rating'] = EloRating::userRating($fixture['player1_id'], $this->data['tournament_id']);
			$fixture['player2_rating'] = EloRating::userRating($fixture['player2_id'], $this->data['tournament_id']);
			$fixture['expected'] = EloRating::expected($fixture['player1_rating'], $fixture['player2_rating']);
			$fixture['expected_percent'] = round($fixture['expected'] * 100);
		}
	}

	/**
	 * Main method that execute @property query and binds input data.
	 *
	 * @method main
	 * @protected
	 */
	protected function main() {
		$stmt = Database::$conn->prepare($this->query);

		$stmt->bindParam(':player_id', $this->data['player_id']);
		$stmt->bindParam(':tournament_id', $this->data['tournament_id']);
		$stmt->bindParam(':user_id', $_SESSION['user']['id']);

		if ($stmt->execute()) {
			$this->return_data['fixtures'] = $stmt->fetchAll(PDO::FETCH_ASSOC);
			$this->attachRatings();
		}
		else {
			$this->success = false;
			$this->error_msg = "Failed to execute query";
		}
	}
}

?>\end{lstlisting}
}
\textbf{Login.php}

Path: /models/Login.php
{\scriptsize
\begin{lstlisting}
<?php

require_once(dirname(__DIR__) . '/php/Model.php');
require_once(dirname(__DIR__) . '/php/Database.php');

session_start();


/**
 * Login model for logging into the system.
 * 
 * @class Login
 * @extends Model
 */
/*&
 * @param email {String} Email of the user.
 * @param password {String} Password of the user.
 */
class Login extends Model
{
	/**
	 * SQL query string for getting the details of the user.
	 * Finds user by email.
	 *
	 * @property query
	 * @type String
	 */
	private $query = <<<SQL
		SELECT id, email, first_name, last_name, hash
		FROM users
		WHERE email=:email
SQL;

	/**
	 * Connects to the database.
	 * Binds query and parameters.
	 *
	 * @property stmt
	 * @type Object
	 * @private
	 */
	private $stmt;
	/**
	 * Stores result from database.
	 * User data.
	 *
	 * @property user
	 * @type Object
	 * @private
	 */
	private $user = array();


	/**
	 * Stores the user details in the current user session.
	 *
	 * @method storeSession
	 * @private
	 */
	private function storeSession() {
		$_SESSION['user'] = array();
		$_SESSION['user']['id'] = $this->user['id'];
		$_SESSION['user']['email'] = $this->user['email'];
		$_SESSION['user']['first_name'] = $this->user['first_name'];
		$_SESSION['user']['last_name'] = $this->user['last_name'];
	}

	/**
	 * Checks to see if password entered matches the password associated to the email entered.
	 *
	 * @method verify
	 * @private
	 */
	private function verify() {
		$this->user = $this->stmt->fetchAll(PDO::FETCH_ASSOC)[0];

		// check if password matches
		if (password_verify($this->data['password'], $this->user['hash'])) {
			$this->storeSession();
		}
		else {
			$this->success = false;
			$this->error_msg = 'Incorrect email or password';
		}
	}


	/**
	 * Defines @property result.
	 * Binds parameters to @property query.
	 *
	 * @method main
	 * @protected
	 */
	protected function main() {
		$this->stmt = Database::$conn->prepare($this->query);

		$this->stmt->bindParam(':email', $this->data['email']);
		
		// check if query is successful
		if ($this->stmt->execute()) {
			$this->verify();
		}
		else {
			$this->success = false;
			$this->error_msg = 'Failed to execute query';
		}
	}
}

?>\end{lstlisting}
}
\textbf{Logout.php}

Path: /models/Logout.php
{\scriptsize
\begin{lstlisting}
<?php

require_once(dirname(__DIR__) . '/php/Model.php');

session_start();


/**
 * Logout model called when the user logs out.
 *
 * @class Logout
 * @extends Model
 */
/*&
 */
class Logout extends Model
{
	/**
	 * Method that destroys the session.
	 *
	 * @method main
	 * @protected
	 */
	protected function main() {
		session_unset();
		session_destroy();
	}
}

?>\end{lstlisting}
}
\textbf{Register.php}

Path: /models/Register.php
{\scriptsize
\begin{lstlisting}
<?php

require_once(dirname(__DIR__) . '/php/Model.php');
require_once(dirname(__DIR__) . '/php/Database.php');
require_once(dirname(__DIR__) . '/superclasses/Validate.php');

session_start();


/**
 * Register model used to register onto the system.
 *
 * @class Register
 * @extends Model
 */
/*&
 * @param email {String} The email of the user.
 * @param first_name {String} The first name of the user.
 * @param last_name {String} The last name of the user.
 * @param password {String} The password of the user.
 *
 * @return id {Integer} Id of the user just registered.
 */
class Register extends Model
{
	/**
	 * SQL query string for inserting user data into the database.
	 *
	 * @property query
	 * @type String
	 * @private
	 */
	private $query = <<<SQL
		INSERT INTO users (email, first_name, last_name, hash)
		VALUES (:email, :first_name, :last_name, :hash)
SQL;

	/**
	 * SQL query string for checking whether email is already in use.
	 *
	 * @property query_email
	 * @type String
	 * @private
	 */
	private $query_email = <<<SQL
		SELECT COUNT(*) count
		FROM users
		WHERE email = :email
SQL;
	
	/**
	 * For executing the query string.
	 *
	 * @property result
	 * @type Object
	 * @private
	 */
	private $stmt;
	/**
	 * Hash of the password entered.
	 *
	 * @property hash
	 * @type String
	 * @private
	 */
	private $hash;


	/**
	 * Used to bind the parameters to @property query.
	 * Data inputted by user is bound.
	 *
	 * @method bindParams
	 * @private
	 */
	private function bindParams() {
		$this->stmt->bindParam(':email', $this->data['email']);
		$this->stmt->bindParam(':first_name', $this->data['first_name']);
		$this->stmt->bindParam(':last_name', $this->data['last_name']);
		$this->stmt->bindParam(':hash', $this->hash);
	}

	/**
	 * Method that will check whether email is already in use.
	 *
	 * @method checkEmail
	 * @return {Boolean} Whether email is already used.
	 */
	private function checkEmail() {
		$stmt = Database::$conn->prepare($this->query_email);

		$stmt->bindParam(':email', $this->data['email']);

		if ($stmt->execute()) {
			return $stmt->fetchAll(PDO::FETCH_ASSOC)[0]['count'];
		}

		return false;
	}

	/**
	 * Validate email, first name, last name, and password.
	 *
	 * @method validate
	 * @private
	 * @return {Boolean} Whether all is valid.
	 */
	private function validate() {
		$validateFirstName = Validate::userName($this->data['first_name'], 'First name');
		$validateLastName = Validate::userName($this->data['last_name'], 'Last name');
		$validateEmail = Validate::email($this->data['email']);
		$validatePassword = Validate::password($this->data['password']);

		if (!$validateFirstName['success']) {
			$this->error_msg = $validateFirstName['error_msg'];

			return false;
		}
		if (!$validateLastName['success']) {
			$this->error_msg = $validateLastName['error_msg'];

			return false;
		}
		if (!$validateEmail['success']) {
			$this->error_msg = $validateEmail['error_msg'];

			return false;
		}
		if ($this->checkEmail()) {
			$this->error_msg = "Email already in use";

			return false;
		}
		if (!$validatePassword['success']) {
			$this->error_msg = $validatePassword['error_msg'];

			return false;
		}

		return true;
	}

	/**
	 * Prepares query.
	 * Hashes password.
	 * Executes query.
	 *
	 * @method main
	 * @public
	 */
	public function main() {
		if ($this->validate()) {
			$this->stmt = Database::$conn->prepare($this->query);
			$this->hash = password_hash($this->data['password'], PASSWORD_BCRYPT);

			$this->bindParams();

			if (!$this->stmt->execute()) {
				$this->success = false;
			}

			$this->return_data['id'] = Database::$conn->lastInsertId($stmt);
		}
		else {
			$this->success = false;
		}
	}
}

?>\end{lstlisting}
}
\textbf{ResultDelete.php}

Path: /models/ResultDelete.php
{\scriptsize
\begin{lstlisting}
<?php

require_once(dirname(__DIR__) . '/php/Model.php');
require_once(dirname(__DIR__) . '/php/Database.php');
require_once(dirname(__DIR__) . '/superclasses/Tournament.php');

session_start();


/**
 * Model that is called to delete a particular result.
 *
 * @method ResultDelete
 * @extends Tournament
 */
/*&
 * @param id {Integer} Id of the result to delete.
 */
class ResultDelete extends Tournament
{
	/**
	 * SQL query string for deleting a result.
	 * Also deletes the users' association with the result.
	 *
	 * @property query
	 * @type String
	 * @private
	 */
	private $query = <<<SQL
		DELETE r.*, ru.*
		FROM results r
		INNER JOIN result_user_maps ru
		ON ru.result_id = r.id
		WHERE r.id = :id
SQL;

	/**
	 * Data associated with the result that the user wants to delete.
	 *
	 * @property result_data
	 * @type Array
	 * @private
	 */
	private $result_data;


	/**
	 * Method that actually deletes the result.
	 *
	 * @method delete
	 * @private
	 */
	private function delete() {
		$stmt = Database::$conn->prepare($this->query);

		$stmt->bindParam(':id', $this->data['id']);

		if (!$stmt->execute()) {
			$this->success = false;
			$this->error_msg = "Failed to execute SQL query";
		}
	}

	/**
	 * Validates whether the user can delete the result.
	 *
	 * @method validate
	 * @private
	 */
	private function validate() {
		// Check if user is logged in.
		if (!isset($_SESSION['user'])) {
			$this->error_msg = "You must be logged in";

			return false;
		}
		// Check if user is league manager.
		if ($this->isLeagueManager($_SESSION['user']['id'], $this->result_data['tournament_id'])) {
			return true;
		}
		// If not, check if it is the user's result.
		else if ($this->result_data['player1_id'] == $_SESSION['user']['id'] || $this->result_data['player2_id'] == $_SESSION['user']['id']) {
			return true;
		}
		else {
			$this->error_msg = "You don't have permission";

			return false;
		}

		return true;
	}

	/**
	 * Method that gets the result data from @class ResultGet.
	 *
	 * @method getResult
	 * @private
	 */
	private function getResult() {
		$ResultGet = new ResultGet(true);

		$this->result_data = $ResultGet->call(array(
			'result_id' => $this->data['id']
		))['results'][0];
	}

	/**
	 * Main method.
	 *
	 * @method main
	 * @protected
	 */
	protected function main() {
		$this->getResult();

		if ($this->validate()) {
			$this->delete();
		}
		else {
			$this->success = false;
		}
	}
}

?>\end{lstlisting}
}
\textbf{ResultEnter.php}

Path: /models/ResultEnter.php
{\scriptsize
\begin{lstlisting}
<?php

require_once(dirname(__DIR__) . '/php/Model.php');
require_once(dirname(__DIR__) . '/php/Database.php');
require_once(dirname(__DIR__) . '/superclasses/Tournament.php');
require_once(dirname(__DIR__) . '/superclasses/EloRating.php');
require_once(dirname(__DIR__) . '/superclasses/ResultMethods.php');

session_start();


/**
 * Model used for entering a result.
 *
 * @class ResultEnter
 * @extends Tournament
 */
/*&
 * @param tournament_id {Integer} Id of the tournament.
 * @param player1_id {Integer} Id of player 1.
 * @param player2_id {Integer} Id of player 2.
 * @param player1_score {Integer} Player 1's score.
 * @param player2_score {Integer} Player 2's score.
 */
class ResultEnter extends Tournament
{
	/**
	 * SQL query string for inserting result info.
	 *
	 * @property query_info
	 * @type String
	 * @private
	 */
	private $query_info = <<<SQL
		INSERT INTO results (tournament_id, date)
		VALUES (:tournament_id, :date)
SQL;
	/**
	 * SQL query string for inserting result score.
	 *
	 * @property query_score
	 * @type String
	 * @private
	 */
	private $query_score = <<<SQL
		INSERT INTO result_user_maps (result_id, user_id, score, rating, rating_change)
		VALUES (:result_id, :user_id, :score, :new_rating, :rating_change)
SQL;

	/**
	 * Id of the result that is being entered.
	 *
	 * @property id_of_result
	 * @type Integer
	 * @private
	 */
	private $id_of_result;


	/**
	 * Method that checks whether the entered score is valid.
	 *
	 * @method validScore
	 * @private
	 */
	private function validScore() {
		if (!is_int($this->data['player1_score']) || !is_int($this->data['player2_score'])) {
			return false;
		}

		if ($this->data['player1_score'] >= 0 && $this->data['player1_score'] <= 3 && $this->data['player2_score'] >= 0 && $this->data['player2_score'] <= 3) {
			if (($this->data['player1_score'] != 3 && $this->data['player2_score'] == 3) || ($this->data['player2_score'] != 3 && $this->data['player1_score'] == 3)) {
				return true;
			}
		}

		return false;
	}

	/**
	 * Main validation method.
	 *
	 * @method validate
	 * @private
	 * @return {Boolean} Whether valid to enter result.
	 */
	private function validate() {
		// Check if player 1 is in league.
		if (!$this->isPlayer($this->data['player1_id'], $this->data['tournament_id'])) {
			$this->error_msg = "Player 1 is not in the league";

			return false;
		}		
		// Check if player 2 is in league.
		if (!$this->isPlayer($this->data['player2_id'], $this->data['tournament_id'])) {
			$this->error_msg = "Player 2 is not in the league";

			return false;
		}

		// Check if player 1 is not player 2.
		if ($this->data['player1_id'] == $this->data['player2_id']) {
			$this->error_msg = "You can't enter a result against the same player";

			return false;
		}

		// Check if user is one of the players.
		if ($_SESSION['user']['id'] != $this->data['player1_id'] &&
			$_SESSION['user']['id'] != $this->data['player2_id'] &&
			!$this->isLeagueManager($_SESSION['user']['id'], $this->data['tournament_id'])) {
			$this->error_msg = "You must be a player or league manager to enter this result";

			return false;
		}
		
		// Check if result already exists.
		if (ResultMethods::resultExists($this->data['player1_id'], $this->data['player2_id'], $this->data['tournament_id'])) {
			$this->error_msg = "Result already exists";

			return false;
		}
		// Check if score is valid.
		if (!$this->validScore()) {
			$this->error_msg = "Not a valid score";

			return false;
		}

		return true;
	}

	/**
	 * Method for inserting the score of a player.
	 *
	 * @method insertScore
	 * @param user_id {Integer} Id of the user whose score to input.
	 * @param score {Integer} Score that the user achieved.
	 * @param new_rating {Integer} New rating of the user.
	 * @param rating_change {Integer} Rating change.
	 * @private
	 */
	private function insertScore($user_id, $score, $new_rating, $rating_change) {
		$stmt = Database::$conn->prepare($this->query_score);

		$stmt->bindParam(':result_id', $this->id_of_result);
		$stmt->bindParam(':user_id', $user_id);
		$stmt->bindParam(':score', $score);
		$stmt->bindParam(':new_rating', $new_rating);
		$stmt->bindParam(':rating_change', $rating_change);

		$stmt->execute();
	}

	/**
	 * Method used to insert general result info.
	 *
	 * @method insertInfo
	 * @private
	 */
	private function insertInfo() {
		$stmt = Database::$conn->prepare($this->query_info);

		$stmt->bindParam(':tournament_id', $this->data['tournament_id']);
		// date is format yymmdd_hhmmss
		$stmt->bindParam(':date', date('ymdHis'));

		if ($stmt->execute()) {
			$this->id_of_result = Database::$conn->lastInsertId();
		}
		else {
			$this->success = false;
			$this->error_msg = "Failed to execute query";
		}
	}

	/**
	 * Calls all general methods.
	 *
	 * @method general
	 * @private
	 */
	private function general() {
		if ($this->validate()) {
			$player1_elo = new EloRating($this->data['player1_id'], $this->data['player2_id'], $this->data['tournament_id'], $this->data['player1_score'], $this->data['player2_score']);
			$player2_elo = new EloRating($this->data['player2_id'], $this->data['player1_id'], $this->data['tournament_id'], $this->data['player2_score'], $this->data['player1_score']);

			$this->insertInfo();
			$this->insertScore($this->data['player1_id'], $this->data['player1_score'], $player1_elo->new_rating, $player1_elo->rating_change);
			$this->insertScore($this->data['player2_id'], $this->data['player2_score'], $player2_elo->new_rating, $player2_elo->rating_change);
		}
		else {
			$this->success = false;
		}
	}

	/**
	 * Checks login.
	 *
	 * @method main
	 * @protected
	 */
	protected function main() {
		if (isset($_SESSION['user'])) {
			$this->general();
		}
		else {
			$this->success = false;
			$this->error_msg = "You must be logged in";
		}
	}
}

?>\end{lstlisting}
}
\textbf{ResultGet.php}

Path: /models/ResultGet.php
{\scriptsize
\begin{lstlisting}
<?php

require_once(dirname(__DIR__) . '/php/Database.php');
require_once(dirname(__DIR__) . '/php/Model.php');


/**
 * Model for retrieving information on a particular result.
 *
 * @class ResultGet
 * @extends Model
 */
/*&
 * @param result_id {Integer} Id of the result.
 * @param player1_id {Integer} Id of player 1.
 * @param player2_id {Integer} Id of player 2.
 * @param tournament_id {Integer} Id of tournament.
 *
 * @return results {Array}
 *   @return [].id {Integer} Id of the result.
 *   @return [].tournament_id {Integer} Id of the tournament.
 *   @return [].tournament_name {String} Name of the tournament.
 *   @return [].date {String} Date-time the match was played.
 *   @return [].player1_id {Integer} Id of player 1.
 *   @return [].player2_id {Integer} Id of player 2.
 *   @return [].player1_name {String} Name of player 1.
 *   @return [].player2_name {String} Name of player 2.
 *   @return [].player1_rating {Integer} Player 1 rating before the match.
 *   @return [].player2_rating {Integer} Player 2 rating before the match.
 *   @return [].player1_rating_change {Integer} Player 1 rating change.
 *   @return [].player2_rating_change {Integer} Player 2 rating change.
 *   @return [].score1 {Integer} Score of player 1.
 *   @return [].score2 {Integer} Score of player 2.
 */
class ResultGet extends Model
{
	/**
	 * SQL query string for getting result information.
	 *
	 * @property query
	 * @type String
	 * @private
	 */
	private $query = <<<SQL
		SELECT
		r.id,
		r.tournament_id,
		t.name tournament_name,
		r.date,
		ru1.user_id player1_id,
		ru2.user_id player2_id,
		CONCAT(u1.first_name, ' ', u1.last_name) player1_name,
		CONCAT(u2.first_name, ' ', u2.last_name) player2_name,
		ru1.rating - ru1.rating_change player1_rating,
		ru2.rating - ru2.rating_change player2_rating,
		ru1.rating_change player1_rating_change,
		ru2.rating_change player2_rating_change,
		ru1.score score1,
		ru2.score score2
		FROM result_user_maps ru1

		INNER JOIN result_user_maps ru2
		ON ru2.result_id = ru1.result_id
		INNER JOIN results r
		ON r.id = ru1.result_id
		
		INNER JOIN users u1
		ON ru1.user_id = u1.id
		INNER JOIN users u2
		ON ru2.user_id = u2.id
		INNER JOIN tournaments t
		ON r.tournament_id = t.id
		INNER JOIN tournament_user_maps tu1
		ON tu1.user_id = u1.id AND tu1.tournament_id = t.id
		INNER JOIN tournament_user_maps tu2
		ON tu2.user_id = u2.id AND tu2.tournament_id = t.id

		WHERE
		ru1.user_id <> ru2.user_id AND
		CASE WHEN :player1_id IS NULL AND :player2_id IS NULL THEN ru1.user_id > ru2.user_id ELSE TRUE END AND
		r.id = CASE WHEN :result_id IS NULL THEN r.id ELSE :result_id END AND
		ru1.user_id = CASE WHEN :player1_id IS NULL THEN ru1.user_id ELSE :player1_id END AND
		ru2.user_id = CASE WHEN :player2_id IS NULL THEN ru2.user_id ELSE :player2_id END AND
		tu1.is_player = TRUE AND
		tu2.is_player = TRUE AND
		r.tournament_id = CASE WHEN :tournament_id IS NULL THEN r.tournament_id ELSE :tournament_id END

		ORDER BY r.date DESC
SQL;


	/**
	 * Main method that executes @property query.
	 *
	 * @method main
	 * @protected
	 */
	protected function main() {
		$stmt = Database::$conn->prepare($this->query);

		$stmt->bindParam(':result_id', $this->data['result_id']);
		$stmt->bindParam(':player1_id', $this->data['player1_id']);
		$stmt->bindParam(':player2_id', $this->data['player2_id']);
		$stmt->bindParam(':tournament_id', $this->data['tournament_id']);

		if ($stmt->execute()) {
			$this->return_data['results'] = array();
			$this->return_data['results'] = $stmt->fetchAll(PDO::FETCH_ASSOC);
		}
		else {
			$this->error_msg = "Failed to execute query";
			$this->success = false;
		}
	}
}

?>\end{lstlisting}
}
\textbf{Status.php}

Path: /models/Status.php
{\scriptsize
\begin{lstlisting}
<?php

require_once(dirname(__DIR__) . '/php/Database.php');
require_once(dirname(__DIR__) . '/php/Model.php');

session_start();


/**
 * Model for getting the status of the current user.
 *
 * @class Status
 * @extends Model
 */
/*&
 * @return logged_in {Boolean} Whether you are logged in.
 * @return user {Array} User data if logged in:
 *  @return user.id {Integer} Your id.
 *  @return user.email {String} Your email address.
 *  @return user.first_name {String} Your first name.
 *  @return user.last_name {String} Your last name.
 *
 */
class Status extends Model
{
	/**
	 * Method that stores user data in @property return_data.
	 *
	 * @method logged_in
	 * @private
	 */
	private function logged_in() {
		$this->return_data['user'] = $_SESSION['user'];
	}

	/**
	 * Method that checks if the user is logged in.
	 * Then sets @property return_data.
	 *
	 * @method main
	 * @protected
	 */
	protected function main() {
		$this->return_data['logged_in'] = false;

		// if logged in
		if (isset($_SESSION['user'])) {
			$this->return_data['logged_in'] = true;
			$this->logged_in();
		}
	}
}

?>\end{lstlisting}
}
\textbf{TournamentCreate.php}

Path: /models/TournamentCreate.php
{\scriptsize
\begin{lstlisting}
<?php

require_once(dirname(__DIR__) . '/superclasses/Tournament.php');
require_once(dirname(__DIR__) . '/superclasses/Validate.php');

session_start();


/**
 * Model for users to create tournaments.
 *
 * @class TournamentCreate
 * @extends Tournament
 */
/*&
 * @param name {String} The name of the tournament.
 * @param description {String} The description of the tournament.
 *
 * @return id {Integer} Id of the tournament just created.
 */
class TournamentCreate extends Tournament
{
	/**
	 * SQL query string for creating a tournament.
	 *
	 * @property query
	 * @type String
	 * @private
	 */
	private $query = <<<SQL
		INSERT INTO tournaments (name, description)
		VALUES (:name, :description)
SQL;
	/**
	 * Query for making the creator of the league a league manager.
	 *
	 * @property query_add_league_manager
	 * @type String
	 * @private
	 */
	private $query_add_league_manager = <<<SQL
		INSERT INTO tournament_user_maps (tournament_id, user_id, is_league_manager, is_player)
		VALUES (:tournament_id, :user_id, 1, 0)
SQL;

	/**
	 * Statement object for executing @property query.
	 *
	 * @property stmt
	 * @type Object
	 * @private
	 */
	private $stmt;

	/**
	 * Id of the tournament that has just been created.
	 *
	 * @property tournament_id
	 * @private
	 */
	private $tournament_id;


	/**
	 * Method which returns the tournament data.
	 * Data collected from @class TournamentGet.
	 *
	 * @method returnTournamentData
	 * @private
	 */
	private function returnTournamentData() {
		$this->return_data['id'] = $this->tournament_id;
	}

	/**
	 * Method for attaching the current user as the league manager.
	 *
	 * @method addLeagueManager
	 * @private
	 */
	private function addLeagueManager() {
		$stmt = Database::$conn->prepare($this->query_add_league_manager);

		$stmt->bindParam(':user_id', $_SESSION['user']['id']);
		// Bind the tournament id as the last insert id.
		$stmt->bindParam(':tournament_id', $this->tournament_id);

		if (!$stmt->execute() || $stmt->rowCount() != 1) {
			$this->error_msg = "Failed to add you as a league manager";
			$this->success = false;
		}
		else {
			$this->returnTournamentData();
		}
	}

	/**
	 * Main function for creating database object, binding params, and executing query.
	 *
	 * @method create
	 * @private
	 */
	private function create() {
		$this->stmt = Database::$conn->prepare($this->query);

		$this->stmt->bindParam(':name', $this->data['name']);
		$this->stmt->bindParam(':description', $this->data['description']);

		if (!$this->stmt->execute()) {
			$this->error_msg = "Failed to execute query";
			$this->success = false;
		}
		else {
			// Sets @property tournament_id
			$this->tournament_id = Database::$conn->lastInsertId();

			// Add league manager if didn't fail.
			$this->addLeagueManager();
		}
	}

	/**
	 * Main function for checking whether user is logged in.
	 *
	 * @method main
	 * @protected
	 */
	protected function main() {
		if (isset($_SESSION['user'])) {
			$validateName = Validate::tournamentName($this->data['name']);
			$validateDescription = Validate::tournamentDescription($this->data['description']);

			if (!$validateName['success']) {
				$this->error_msg = $validateName['error_msg'];
				$this->success = false;
			}
			else if (!$validateDescription['success']) {
				$this->error_msg = $validateDescription['error_msg'];
				$this->success = false;
			}
			else {
				$this->create();
			}
		}
		else {
			$this->error_msg = "You must be logged in";
			$this->success = false;
		}
	}
}

?>\end{lstlisting}
}
\textbf{TournamentDelete.php}

Path: /models/TournamentDelete.php
{\scriptsize
\begin{lstlisting}
<?php

require_once(dirname(__DIR__) . '/superclasses/Tournament.php');

session_start();


/**
 * Class that deletes a tournament.
 *
 * @class TournamentDelete
 * @extends Tournament
 */
/*&
 * @param id {Integer} Id of the tournament
 */
class TournamentDelete extends Tournament
{
	/**
	 * SQL query string for deleting the league.
	 * Also deletes all associations with the league.
	 * This includes results.
	 *
	 * @property query_league
	 * @type String
	 * @private
	 */
	private $query_league = <<<SQL
		DELETE t.*, tu.*, ru.*, r.*
		FROM tournaments t
		LEFT JOIN tournament_user_maps tu
		ON t.id = tu.tournament_id
		LEFT JOIN results r
		ON r.tournament_id = t.id
		LEFT JOIN result_user_maps ru
		ON ru.result_id = r.id
		WHERE t.id = :id
SQL;

	/**
	 * Method that uses @property query_league to delete the league.
	 *
	 * @method delete
	 * @private
	 */
	private function delete() {
		$stmt = Database::$conn->prepare($this->query_league);

		$stmt->bindParam(':id', $this->data['id']);

		if (!$stmt->execute()) {
			$this->success = false;
			$this->error_msg = "Failed to delete the league";
		}
	}

	/**
	 * Checks whether the user can delete the league.
	 *
	 * @method verify
	 * @private
	 * @return {Boolean} Whether the league is deletable.
	 */
	private function verify() {
		if (!isset($_SESSION['user'])) {
			$this->error_msg = "You must be logged in";

			return false;
		}
		if (!$this->isLeagueManager($_SESSION['user']['id'], $this->data['id'])) {
			$this->error_msg = "You must be a league manager";

			return false;
		}

		return true;
	}

	/**
	 * Result of @method verify determines whether it calls @method delete.
	 *
	 * @method main
	 * @protected
	 */
	protected function main() {
		if ($this->verify()) {
			$this->delete();
		}
		else {
			$this->success = false;
		}
	}
}

?>\end{lstlisting}
}
\textbf{TournamentGet.php}

Path: /models/TournamentGet.php
{\scriptsize
\begin{lstlisting}
<?php

require_once(dirname(__DIR__) . '/superclasses/Tournament.php');

session_start();


/**
 * Model that fetches tournament data based on id.
 *
 * @class TournamentGet
 * @extends Tournament
 */
/*&
 * @param id {Integer} Id of the tournament.
 *
 * @return id {Integer} Id of the tournament.
 * @return name {String} Name of the tournament.
 * @return description {String} Description of the tournament.
 * @return is_league_manager {Boolean} Whether the user calling is a league manager.
 * @return league_managers {Array} Array of league managers.
 *  @return league_managers[].id {Integer} Id of the league manager.
 *  @return league_managers[].first_name {String} First name of the league manager.
 *  @return league_managers[].last_name {String} Last name of the league manager.
 * @return players {Array} Array of players in the tournament.
 *  @return players[].id {Integer} Id of the player.
 *  @return players[].first_name {String} First name of the player.
 *  @return players[].last_name {String} Last name of the player.
 */
class TournamentGet extends Tournament
{
	/**
	 * SQL query string for fetching tournament data.
	 *
	 * @property query
	 * @type String
	 * @private
	 */
	private $query = <<<SQL
		SELECT id, name, description
		FROM tournaments
		WHERE id = :id
SQL;
	/**
	 * SQL query string for fetching the players in the tournament.
	 *
	 * @property query_players
	 * @type String
	 * @private
	 */
	private $query_players = <<<SQL
		SELECT u.id, u.first_name, u.last_name
		FROM users u
		INNER JOIN tournament_user_maps tu
		ON tu.user_id = u.id
		WHERE tu.tournament_id = :id AND tu.is_player = true
SQL;
	/**
	 * SQL query string for fetching the league managers of the tournament.
	 *
	 * @property query_leauge_managers
	 * @type String
	 * @private
	 */
	private $query_league_managers = <<<SQL
		SELECT u.id, u.first_name, u.last_name
		FROM users u
		INNER JOIN tournament_user_maps tu
		ON tu.user_id = u.id
		WHERE tu.tournament_id = :id AND tu.is_league_manager = true
SQL;


	/**
	 * Method that queries players in the tournament.
	 *
	 * @method getPlayers
	 * @private
	 */
	private function getPlayers() {
		$stmt = Database::$conn->prepare($this->query_players);

		$stmt->bindParam(':id', $this->data['id']);

		if ($stmt->execute()) {
			$this->return_data['players'] = $stmt->fetchAll(PDO::FETCH_ASSOC);
		}
		else {
			$this->success = false;
		}
	}

	/**
	 * Method that queries league managers in the tournament.
	 *
	 * @method getLeagueManagers
	 * @private
	 */
	private function getLeagueManagers() {
		$stmt = Database::$conn->prepare($this->query_league_managers);

		$stmt->bindParam(':id', $this->data['id']);

		if ($stmt->execute()) {
			$this->return_data['league_managers'] = $stmt->fetchAll(PDO::FETCH_ASSOC);
		}
		else {
			$this->success = false;
		}
	}

	/**
	 * Method that gets the data about the tournament.
	 *
	 * @method getTournamentData
	 * @private
	 */
	private function getTournamentData() {
		$stmt = Database::$conn->prepare($this->query);

		$stmt->bindParam(':id', $this->data['id']);

		if ($stmt->execute()) {
			$this->return_data = $stmt->fetchAll(PDO::FETCH_ASSOC)[0];
		}
		else {
			$this->success = false;
		}


		// Shows whether the current user is a league manager.
		$this->return_data['is_league_manager'] = $this->isLeagueManager($_SESSION['user']['id'], $this->data['id']);
	}

	/**
	 * Method that fetches the database info.
	 *
	 * @method main
	 * @protected
	 * @return {Array} Tournament data.
	 */
	protected function main() {
		if ($this->tournamentExistsId($this->data['id'])) {
			$this->getTournamentData();
			$this->getPlayers();
			$this->getLeagueManagers();
		}
		else {
			$this->success = false;
			$this->error_msg = "Tournament doesn't exist";
		}
	}
}

?>\end{lstlisting}
}
\textbf{TournamentLeagueTable.php}

Path: /models/TournamentLeagueTable.php
{\scriptsize
\begin{lstlisting}
<?php

require_once(dirname(__DIR__) . '/php/Model.php');
require_once(dirname(__DIR__) . '/php/Database.php');
require_once(dirname(__DIR__) . '/superclasses/EloRating.php');
require_once(dirname(__DIR__) . '/models/ResultGet.php');


/**
 * Class for generating a league table.
 *
 * @class TournamentLeagueTable
 * @extends Model
 */
/*&
 * @param id {Integer} If of the tournament.
 *
 * @return table {Array} Array that holds all the players in order.
 *   @return [].user_id {Integer} Id of player.
 *   @return [].name {String} Full name of player.
 *   @return [].played {Integer} Number of games played.
 *   @return [].wins {Integer} Number of games won.
 *   @return [].loses {Integer} Number of games lost.
 *   @return [].points {Integer} Number of points from games played.
 *   @return [].rating {Integer} Rating of player in the specific tournament.
 */
class TournamentLeagueTable extends Model
{
	/**
	 * Array that stores the table data.
	 *
	 * @property table
	 * @private
	 * @type Array
	 */
	private $table = array();

	/**
	 * SQL query string for getting the players in the leauge.
	 *
	 * @property query
	 * @private
	 * @type String
	 */
	private $query = <<<SQL
		SELECT
		u.id user_id,
		CONCAT(u.first_name, ' ', u.last_name) name,
		0 played,
		0 wins,
		0 loses,
		0 points,
		null rating
		FROM tournament_user_maps tu
		INNER JOIN users u
		ON u.id = tu.user_id
		WHERE
		tu.tournament_id = :id AND
		tu.is_player = TRUE
SQL;


	/**
	 * If that player is unpopulated, then populate the table.
	 *
	 * @method populateTable
	 * @private
	 */
	private function populateTable() {
		$stmt = Database::$conn->prepare($this->query);

		$stmt->bindParam(':id', $this->data['id']);

		if ($stmt->execute()) {
			$users = $stmt->fetchAll(PDO::FETCH_ASSOC);

			foreach($users as &$user) {
				$this->table[$user['user_id']] = $user;
			}
		}
		else {
			$this->error_msg = "Failed to execute query";
			$this->success = false;
		}
	}

	/**
	 * Calculate points and put in table.
	 *
	 * @method calcuate
	 * @private
	 */
	private function calculate() {
		$ResultGet = new ResultGet(true);
		$results = $ResultGet->call(array(
			'tournament_id' => $this->data['id']
		))['results'];

		foreach ($results as &$result) {
			$points1 = 0;
			$points2 = 0;
			$this->table[$result['player1_id']]['played'] += 1;
			$this->table[$result['player2_id']]['played'] += 1;

			// Player 1 wins.
			if ($result['score1'] == 3) {
				$points1 = 6 - $result['score2'];
				$points2 = 1 + $result['score2'];
				$this->table[$result['player1_id']]['wins'] += 1;
				$this->table[$result['player2_id']]['loses'] += 1;
			}
			// Player 2 wins.
			else if ($result['score2'] == 3) {
				$points1 = 1 + $result['score1'];
				$points2 = 6 - $result['score1'];
				$this->table[$result['player2_id']]['wins'] += 1;
				$this->table[$result['player1_id']]['loses'] += 1;
			}

			// Add points to total.
			$this->table[$result['player1_id']]['points'] += $points1;
			$this->table[$result['player2_id']]['points'] += $points2;
		}
	}

	/**
	 * Order table and return.
	 *
	 * @method order
	 * @private
	 */
	private function order() {
		// Set table to return data.
		foreach ($this->table as &$user) {
			$this->table[$user['user_id']]['rating'] = EloRating::userRating($user['user_id']);

			array_push($this->return_data['table'], $user);
		}

		// Order table.
		usort($this->return_data['table'], function($a, $b) {
			// If level on points then order by wins.
			if ($a['points'] == $b['points']) {
				return $b['wins'] > $a['wins'];
			}

			// Else order by points.
			return $b['points'] > $a['points'];
		});
	}

	/**
	 * Method the executes @property query.
	 * Returns league table.
	 *
	 * @method main
	 * @protected
	 */
	protected function main() {
		$this->return_data['table'] = array();

		$this->populateTable();
		$this->calculate();
		$this->order();		
	}
}

?>\end{lstlisting}
}
\textbf{TournamentManagerAttach.php}

Path: /models/TournamentManagerAttach.php
{\scriptsize
\begin{lstlisting}
<?php

require_once(dirname(__DIR__) . '/superclasses/TournamentManagerAlter.php');


/**
 * Model for adding a user as a manager of a tournament.
 *
 * @class TournamentManagerAttach
 * @extends TournamentManagerAlter
 */
/*&
 * @param user_id {Integer} Id of the user to make a manager.
 * @param tournament_id {Integer} Id of the tournament.
 */
class TournamentManagerAttach extends TournamentManagerAlter
{
	/**
	 * User becomes a league manager.
	 *
	 * @property is_league_manager
	 * @protected
	 */
	protected $is_league_manager = true;


	/**
	 * Main method calls @method subMain.
	 *
	 * @method main
	 * @protected
	 */
	protected function main() {
		$this->subMain();
	}
}

?>\end{lstlisting}
}
\textbf{TournamentManagerRemove.php}

Path: /models/TournamentManagerRemove.php
{\scriptsize
\begin{lstlisting}
<?php

require_once(dirname(__DIR__) . '/superclasses/TournamentManagerAlter.php');


/**
 * Model for removing a league manager from a tournament.
 *
 * @class TournamentManagerRemove
 * @extends TournamentManagerAlter
 */
/*&
 * @param user_id {Integer} Id of the league manager to remove.
 * @param tournament_id {Integer} Id of the tournament
 */
class TournamentManagerRemove extends TournamentManagerAlter
{
	/**
	 * No longer a league manager.
	 *
	 * @property is_league_manager
	 * @protected
	 */
	protected $is_league_manager = false;


	/**
	 * Main method that calls @method subMain.
	 *
	 * @method main
	 * @protected
	 */
	protected function main() {
		$this->subMain();
	}
}

?>\end{lstlisting}
}
\textbf{TournamentPlayerAttach.php}

Path: /models/TournamentPlayerAttach.php
{\scriptsize
\begin{lstlisting}
<?php

require_once(dirname(__DIR__) . '/superclasses/TournamentPlayerAlter.php');


/**
 * Model for adding a user as a player to a tournament.
 *
 * @class TournamentPlayerAttach
 * @extends TournamentPlayerAlter
 */
/*&
 * @param user_id {Integer} Id of the user to be added to the tournament.
 * @param tournament_id {Integer} Id of the tournament.
 */
class TournamentPlayerAttach extends TournamentPlayerAlter
{
	/**
	 * Player becomes true when adding a player.
	 *
	 * @property is_player
	 * @protected
	 */
	protected $is_player = true;


	/**
	 * Calls @method subMain.
	 *
	 * @method main
	 * @protected
	 */
	protected function main() {
		$this->subMain();
	}
}

?>\end{lstlisting}
}
\textbf{TournamentPlayerRemove.php}

Path: /models/TournamentPlayerRemove.php
{\scriptsize
\begin{lstlisting}
<?php

require_once(dirname(__DIR__) . '/superclasses/TournamentPlayerAlter.php');


/**
 * Model for removing a player from a tournament.
 *
 * @class TournamentPlayerRemove
 * @extends TournamentPlayerAlter
 */
/*&
 * @param user_id {Integer} Id of the user to be removed from the tournament.
 * @param tournament_id {Integer} Id of the tournament.
 */
class TournamentPlayerRemove extends TournamentPlayerAlter
{
	/**
	 * Player becomes false when removing a player.
	 *
	 * @property is_player
	 * @protected
	 */
	protected $is_player = false;


	/**
	 * Calls @method subMain
	 *
	 * @method main
	 * @protected
	 */
	protected function main() {
		$this->subMain();
	}
}

?>\end{lstlisting}
}
\textbf{TournamentSearch.php}

Path: /models/TournamentSearch.php
{\scriptsize
\begin{lstlisting}
<?php

require_once(dirname(__DIR__) . '/php/Model.php');
require_once(dirname(__DIR__) . '/php/Database.php');


/**
 * Model that is used to search for a tournament by name.
 *
 * @class TournamentSearch
 * @extends Model
 */
/*&
 * @param name {String} Search for tournament with similar name.
 *
 * @return tournaments {Array} Array of all tournaments that match.
 *   @return [].id {Integer} Id of the tournament.
 *   @return [].name {String} Name of the tournament.
 */
class TournamentSearch extends Model
{
	/**
	 * SQL query string for searching for tournaments and returning their name and id.
	 *
	 * @property query
	 * @type String
	 * @private
	 */
	private $query = <<<SQL
		SELECT id, name
		FROM tournaments
		WHERE
		name LIKE :name
SQL;


	/**
	 * Main method.
	 * Used to execute @property query.
	 *
	 * @method main
	 * @protected
	 */
	protected function main() {
		// Redefine @param name, SQL syntax for searching with 'LIKE'.
		$this->data['name'] = '%' . $this->data['name'] . '%';

		$stmt = Database::$conn->prepare($this->query);

		$stmt->bindParam(':name', $this->data['name']);

		if ($stmt->execute()) {
			$this->return_data['tournaments'] = $stmt->fetchAll(PDO::FETCH_ASSOC);
		}
		else {
			$this->error_msg = "Failed to execute query";
			$this->success = false;
		}
	}
}

?>\end{lstlisting}
}
\textbf{TournamentUpdate.php}

Path: /models/TournamentUpdate.php
{\scriptsize
\begin{lstlisting}
<?php

require_once(dirname(__DIR__) . '/superclasses/Tournament.php');
require_once(dirname(__DIR__) . '/superclasses/Validate.php');

session_start();


/**
 * Model that updates tournament info.
 *
 * @class TournamentUpdate
 * @extends Tournament
 */
/*&
 * @param id {Integer} Id of tournament.
 * @param name {String} New name of the tournament.
 * @param description {String} New description of the tournament.
 */
class TournamentUpdate extends Tournament
{
	/**
	 * SQL query string that updates tournament info.
	 *
	 * @property query
	 * @type String
	 * @private
	 */
	private $query = <<<SQL
		UPDATE tournaments
		SET name = :name, description = :description
		WHERE id = :id
SQL;


	/**
	 * Validates whether the user can update tournament info.
	 *
	 * @method validate
	 * @private
	 * @return {Boolean} Whether the user has permission.
	 */
	private function validate() {
		$is_league_manager = $this->isLeagueManager($_SESSION['user']['id'], $this->data['id']);
		$tournament_exists = $this->tournamentExistsId($this->data['id']);
		$validateName = Validate::tournamentName($this->data['name']);
		$validateDescription = Validate::tournamentDescription($this->data['description']);

		if (!$tournament_exists) {
			$this->error_msg = "Tournament doesn't exists";

			return false;
		}
		if (!$is_league_manager) {
			$this->error_msg = "You don't have permission to do that";

			return false;
		}
		if (!$validateName['success']) {
			$this->error_msg = $validateName['error_msg'];

			return false;
		}
		if (!$validateDescription['success']) {
			$this->error_msg = $validateDescription['error_msg'];

			return false;
		}

		return true;
	}

	/**
	 * Method that executes @property query.
	 * Thus updating tournament info.
	 *
	 * @method update
	 * @private
	 */
	private function update() {
		$stmt = Database::$conn->prepare($this->query);

		$stmt->bindParam(':id', $this->data['id']);
		$stmt->bindParam(':name', $this->data['name']);
		$stmt->bindParam(':description', $this->data['description']);

		if (!$stmt->execute()) {
			$this->success = false;
			$this->error_msg = "Failed to update tournament";
		}
	}

	/**
	 * Method that checks login, then calls @method update.
	 *
	 * @method main
	 * @protected
	 */
	protected function main() {
		if (isset($_SESSION['user'])) {
			if ($this->validate()) {
				$this->update();
			}
			else {
				$this->success = false;
			}
		}
		else {
			$this->success = false;
			$this->error_msg = "You must be logged in";
		}
	}
}

?>\end{lstlisting}
}
\textbf{UserGet.php}

Path: /models/UserGet.php
{\scriptsize
\begin{lstlisting}
<?php

require_once(dirname(__DIR__) . '/php/Database.php');
require_once(dirname(__DIR__) . '/php/Model.php');
require_once(dirname(__DIR__) . '/superclasses/EloRating.php');

session_start();


/**
 * Model for querying a user based on the id.
 *
 * @class UserGet
 * @extends Model
 */
/*&
 * @param id {Integer} Id of the user to be fetched.
 *
 * @return id {Integer} Id of the user.
 * @return first_name {String} First name of the user.
 * @return last_name {String} Last name of the user.
 * @return rating {Integer} The user's rating.
 * @return leagues_playing {Array}
 *   @return leagues_playing[].id {Integer} Id of league playing in.
 *   @return leagues_playing[].name {String} Name of league playing in.
 * @return  leagues_managing {Array}
 *   @return leagues_managing[].id {Integer} Id of league managing.
 *   @return leagues_managing[].name {String} Name of league managing.
 */
class UserGet extends Model
{
	/**
	 * SQL query string for fetching the user's data.
	 *
	 * @property query
	 * @type String
	 * @private
	 */
	private $query = <<<SQL
		SELECT first_name, last_name, id
		FROM users
		WHERE
		id = :id
SQL;

	/**
	 * SQL query string for getting tournaments that the user is managing.
	 *
	 * @property query_managing
	 * @type String
	 * @private
	 */
	private $query_managing = <<<SQL
		SELECT t.id, t.name
		FROM tournaments t
		INNER JOIN tournament_user_maps tu
		ON tu.tournament_id = t.id
		WHERE
		tu.user_id = :id AND
		tu.is_league_manager = TRUE
SQL;

	/**
	 * SQL query string for getting tournaments that the user is playing in.
	 *
	 * @property query_playing
	 * @type String
	 * @private
	 */
	private $query_playing = <<<SQL
		SELECT t.id, t.name
		FROM tournaments t
		INNER JOIN tournament_user_maps tu
		ON tu.tournament_id = t.id
		WHERE
		tu.user_id = :id AND
		tu.is_player = TRUE
SQL;

	/**
	 * Database object for executing query.
	 *
	 * @property stmt
	 * @type Object
	 * @private
	 */
	private $stmt;


	/**
	 * Method that verifies whether the requested user is the one returned.
	 *
	 * @method verifyResult
	 * @private
	 * @return {Boolean} Whether the result matches request.
	 */
	private function verifyResult() {
		if ($this->return_data['id'] == $this->data['id']) {
			return true;
		}
		else {
			$this->error_msg = "User doesn't exist";
			$this->success = false;

			return false;
		}
	}

	/**
	 * Method that executes the query.
	 *
	 * @method executeQuery
	 * @private
	 * @return {Boolean} Whether executed query successfully.
	 */
	private function executeQuery() {
		if ($this->stmt->execute()) {
			$this->return_data = $this->stmt->fetchAll(PDO::FETCH_ASSOC)[0];

			return true;
		}
		else {
			$this->error_msg = "Failed to execute query";
			$this->success = false;

			return false;
		}
	}

	/**
	 * Get tournament playing in.
	 *
	 * @method getPlaying
	 * @private
	 */
	private function getPlaying() {
		$stmt = Database::$conn->prepare($this->query_playing);

		$stmt->bindParam(':id', $this->data['id']);

		if ($stmt->execute()) {
			$this->return_data['leagues_playing'] = $stmt->fetchAll(PDO::FETCH_ASSOC);
		}
	}

	/**
	 * Method for getting tournaments managing
	 *
	 * @method getManaging
	 * @private
	 */
	private function getManaging() {
		$stmt = Database::$conn->prepare($this->query_managing);

		$stmt->bindParam(':id', $this->data['id']);

		if ($stmt->execute()) {
			$this->return_data['leagues_managing'] = $stmt->fetchAll(PDO::FETCH_ASSOC);
		}
	}

	/**
	 * Method for getting user data.
	 *
	 * @method getUserData
	 * @private
	 * @return {Boolean} Whether user exists
	 */
	private function getUserData() {
		$this->stmt = Database::$conn->prepare($this->query);

		$this->stmt->bindParam(':id', $this->data['id']);

		if ($this->executeQuery()) {
			return $this->verifyResult();
		}
		else {
			return false;
		}
	}

	/**
	 * Main method.
	 * If the user exists, it gets tournament data.
	 *
	 * @method main
	 * @protected
	 */
	protected function main() {
		if ($this->getUserData()) {
			$this->return_data['rating'] = EloRating::userRating($this->data['id']);
			
			$this->getPlaying();
			$this->getManaging();
		}
	}
}

?>\end{lstlisting}
}
\textbf{UserRatings.php}

Path: /models/UserRatings.php
{\scriptsize
\begin{lstlisting}
<?php

require_once(dirname(__DIR__) . '/php/Model.php');
require_once(dirname(__DIR__) . '/php/Database.php');

session_start();


/**
 * Model when called will return all the user's ratings over time.
 *
 * @class UserRatings
 * @extends Model
 */
/*&
 * @param user_id {Integer} Id of the user to query.
 *
 * @return ratings {Array} Array of all the ratings over time.
 *   @return [].rating {Integer} Rating at that moment in time.
 *   @return [].rating_change {Integer} Change of rating.
 *   @return [].date {String} Date and time when rating was the value.
 */
class UserRatings extends Model
{
	/**
	 * SQL query string that fetches all the user's ratings over time.
	 *
	 * @property query
	 * @type String
	 * @private
	 */
	private $query = <<<SQL
		SELECT
		ru.rating,
		ru.rating_change,
		r.date
		FROM result_user_maps ru
		INNER JOIN results r
		ON r.id = ru.result_id
		WHERE ru.user_id = :id
		ORDER BY r.date DESC
SQL;


	/**
	 * Method that executes @property query.
	 *
	 * @method general
	 * @private
	 */
	private function general() {
		$stmt = Database::$conn->prepare($this->query);

		$stmt->bindParam(':id', $this->data['user_id']);

		if ($stmt->execute()) {
			$this->return_data['ratings'] = $stmt->fetchAll(PDO::FETCH_ASSOC);
		}
		else {
			$this->error_msg = "Failed to execute query";
			$this->success = false;
		}
	}

	/**
	 * Checks whether user is logged in then calls @method general.
	 *
	 * @method main
	 * @protected
	 */
	protected function main() {
		if (isset($_SESSION['user'])) {
			$this->general();
		}
		else {
			$this->error_msg = "You must be logged in";
			$this->success = false;
		}
	}
}

?>\end{lstlisting}
}
\textbf{UserSearch.php}

Path: /models/UserSearch.php
{\scriptsize
\begin{lstlisting}
<?php

require_once(dirname(__DIR__) . '/php/Database.php');
require_once(dirname(__DIR__) . '/php/Model.php');


/**
 * Model that is used to search for a user.
 *
 * @class UserSearch
 * @extends Model
 */
/*&
 * @param name {String} Search for user with similar name.
 *
 * @return users {Array} Array of all users that match.
 *    @return [].id {Integer} Id of the user.
 *    @return [].name {String} Full name of the user.
 */
class UserSearch extends Model
{
	/**
	 * SQL query string for searching for users and returning their name and id.
	 *
	 * @property query
	 * @type String
	 * @private
	 */
	private $query = <<<SQL
		SELECT id, CONCAT(first_name, ' ', last_name) name
		FROM users
		WHERE
		CONCAT(first_name, ' ', last_name) LIKE :name
SQL;


	/**
	 * Main method.
	 * Used to execute @property query.
	 *
	 * @method main
	 * @protected
	 */
	protected function main() {
		// Redefine @param name, SQL syntax for searching with 'LIKE'.
		$this->data['name'] = '%' . $this->data['name'] . '%';

		$stmt = Database::$conn->prepare($this->query);
		
		$stmt->bindParam(':name', $this->data['name']);

		if ($stmt->execute()) {
			$this->return_data['users'] = $stmt->fetchAll(PDO::FETCH_ASSOC);
		}
		else {
			$this->error_msg = "Failed to execute query";
			$this->success = false;
		}
	}
}

?>
\end{lstlisting}
}
\textbf{UserUpdate.php}

Path: /models/UserUpdate.php
{\scriptsize
\begin{lstlisting}
<?php

require_once(dirname(__DIR__) . '/php/Model.php');
require_once(dirname(__DIR__) . '/php/Database.php');
require_once(dirname(__DIR__) . '/superclasses/Validate.php');

session_start();


/**
 * This model is used to update user details.
 *
 * @class UserUpdate
 * @extends Model
 */
/*&
 * @param first_name {String} The new first name of the user.
 * @param last_name {String} The new last name of the user.
 * @param home_phone {String} The new home phone number of the user.
 * @param mobile_phone {String} THe new mobile phone number of the user.
 */
class UserUpdate extends Model
{
	/**
	 * SQL query string to update the user's data.
	 *
	 * @property query
	 * @type String
	 * @private
	 */
	private $query = <<<SQL
		UPDATE users
		SET
		first_name = CASE WHEN :first_name IS NULL THEN first_name ELSE :first_name END,
		last_name = CASE WHEN :last_name IS NULL THEN last_name ELSE :last_name END,
		home_phone = :home_phone,
		mobile_phone = :mobile_phone
		WHERE id = :id
SQL;

	/**
	 * Database object variable.
	 *
	 * @property stmt
	 * @type Object
	 * @private
	 */
	private $stmt;


	/**
	 * Main method once @method main has checked login.
	 *
	 * @method update
	 * @private
	 */
	private function update() {
		$this->stmt = Database::$conn->prepare($this->query);

		$this->stmt->bindParam(':id', $_SESSION['user']['id']);
		$this->stmt->bindParam(':first_name', $this->data['first_name']);
		$this->stmt->bindParam(':last_name', $this->data['last_name']);
		$this->stmt->bindParam(':home_phone', $this->data['home_phone']);
		$this->stmt->bindParam(':mobile_phone', $this->data['mobile_phone']);

		if (!$this->stmt->execute()) {
			$this->success = false;
		}
	}

	/**
	 * Method that validates the inputs.
	 * Validating first name, last name, and phone numbers.
	 *
	 * @method validate
	 * @private
	 * @return {Boolean} Whether valid.
	 */
	private function validate() {
		$validateFirstName = Validate::userName($this->data['first_name'], 'First name');
		$validateLastName = Validate::userName($this->data['last_name'], 'Last name');
		$validateHomePhone = Validate::phoneNumber($this->data['home_phone']);
		$validateMobilePhone = Validate::phoneNumber($this->data['mobile_phone']);

		if (!$validateFirstName['success']) {
			$this->error_msg = $validateFirstName['error_msg'];

			return false;
		}
		if (!$validateLastName['success']) {
			$this->error_msg = $validateLastName['error_msg'];

			return false;
		}
		if (!$validateHomePhone['success'] && isset($this->data['home_phone'])) {
			$this->error_msg = $validateHomePhone['error_msg'];

			return false;
		}
		if (!$validateMobilePhone['success'] && isset($this->data['mobile_phone'])) {
			$this->error_msg = $validateMobilePhone['error_msg'];

			return false;
		}

		return true;
	}

	/**
	 * Checks to see if the user is logged in.
	 * Calls @method update if true.
	 *
	 * @method main
	 * @protected
	 */
	protected function main() {
		if (isset($_SESSION['user']) && $this->validate()) {
			$this->update();
		}
		else {
			$this->success = false;
		}
	}
}

?>\end{lstlisting}
}
\subsubsection{Important Classes}
\textbf{API.php}

Path: /php/API.php
{\scriptsize
\begin{lstlisting}
<?php

/**
 * Allows the JavaScript to communicate with the PHP models.
 *
 * @class API
 */
class API
{
	/**
	 * Name of the model to be called.
	 *
	 * @property modelName
	 * @private
	 * @type String
	 */
	private $modelName;


	/**
	 * Creates an instance of the class requested.
	 * Therefore calling the model.
	 *
	 * @method execute
	 * @private
	 */
	private function execute() {
		$model = new $this->modelName();
	}

	/**
	 * Gets the data from GET or POST.
	 *
	 * @method getData
	 * @private
	 */
	private function getData() {
		if (isset($_POST)) {
			$this->modelName = $_POST['model'];
		}
	}

	/**
	 * Method that requires all the files.
	 *
	 * @method requireAll
	 * @private
	 */
	private function requireAll() {
		// Require all models.
		foreach (scandir(dirname(__DIR__) . '/models/') as $filename) {
			$path = dirname(__DIR__) . '/models/' . $filename;

			if (is_file($path)) {
				require_once($path);
			}
		}
	}


	/**
	 * Method that is executed when the API is called.
	 * Calls all of the other methods.
	 *
	 * @method __construct
	 * @public
	 */
	public function __construct() {
		$this->requireAll();
		$this->getData();
		$this->execute();
	}
}

$API = new API();

?>\end{lstlisting}
}
\textbf{Database.php}

Path: /php/Database.php
{\scriptsize
\begin{lstlisting}
<?php

/**
 * The Database class connects the the database when the file is included.
 *
 * @class Database
 */
class Database
{
	/**
	 * This variable is the connection variable for connecting to the database.
	 *
	 * @property conn
	 * @type Object
	 * @static
	 */
	public static $conn;
	/**
	 * SQL query string for deleting all tables
	 *
	 * @property query_delete
	 * @type String
	 * @private
	 * @static
	 */
	private static $query_delete = <<<SQL
		DROP TABLE `results`, `result_user_maps`, `tournaments`, `tournament_user_maps`, `users`
SQL;


	/**
	 * This method is used to create the tables and columns in the database.
	 * The `database.sql` file provides the SQL query string to do this.
	 *
	 * @method create
	 * @public
	 * @static
	 * @return {Boolean} Whether success.
	 */
	public static function create() {
		$query = file_get_contents(dirname(__DIR__) . '/database.sql');
		$stmt = self::$conn->prepare($query);

		return $stmt->execute();
	}

	/**
	 * Method that deletes all the tables in the database.
	 *
	 * @method delete
	 * @public
	 * @static
	 * @return {Boolean} Whether success.
	 */
	public static function delete() {
		$stmt = self::$conn->prepare(self::$query_delete);

		$stmt->execute();

		// Issues when with this query because it doesn't change any rows.
		return true;
	}

	/**
	 * Resets database.
	 * Deletes then recreates.
	 *
	 * @method reset
	 * @public
	 * @static
	 * @return {Boolean} Whether success.
	 */
	public static function reset() {
		$delete_success = self::delete();
		$create_success = self::create();

		return $delete_success && $create_success;
	}

	/**
	 * This method is called when the file is included.
	 * It is used to connect to the database.
	 * Also to set @property conn to new PDO.
	 *
	 * @method init
	 * @public
	 * @static
	 */
	public static function init() {
		$string = file_get_contents(dirname(__DIR__) . '/env.json');
		$data = json_decode($string, true);
		$connData = $data['database'];

		self::$conn = new PDO('mysql:host=' . $connData['servername'] . ';dbname=' . $connData['database'], $connData['username'], $connData['password']);
	}
}


// @method init called straight away.
Database::init();

?>\end{lstlisting}
}
\textbf{Model.php}

Path: /php/Model.php
{\scriptsize
\begin{lstlisting}
<?php

/**
 * Model class, the class is extended by all models.
 * It provides a foundation for all models.
 *
 * @class Model
 */
class Model
{
	/**
	 * The name of the data variable in POST.
	 *
	 * @property name
	 * @private
	 */
	private $name = 'JSON';

	/**
	 * The data object is used to store.
	 *
	 * @property data
	 * @type Array
	 * @protected
	 */
	protected $data = array();

	/**
	 * The object for holding all data that wants to be returned.
	 *
	 * @property return_data
	 * @type Array
	 * @protected
	 */
	protected $return_data;

	/**
	 * Whether model executed successfully.
	 *
	 * @property success
	 * @type {Boolean} Default true.
	 * @protected
	 */
	protected $success;

	/**
	 * Error message string, if error.
	 *
	 * @property error_msg
	 * @type String
	 * @protected
	 */
	protected $error_msg;


	/**
	 * Method that assembles the return object.
	 *
	 * @method returnObj
	 * @private
	 * @return {Array} Return object.
	 */
	private function returnObj() {
		$obj = array();

		if ($this->success) {
			$obj = $this->return_data;
		}
		else {
			$obj['error_msg'] = $this->error_msg;
		}

		$obj['success'] = $this->success;

		return $obj;
	}

	/**
	 * This method is used to decode the JSON data in the post.
	 *
	 * @method
	 * @private
	 * @return {Object} The decoded JSON.
	 */
	private function decodePost() {
		if (isset($_POST[$this->name])) {
			return json_decode($_POST[$this->name], true);
		}
	}

	/**
	 * This returns whether the post data variable is set.
	 *
	 * @method isPost
	 * @private
	 * @return {Boolean} Whether it's set.
	 */
	private function isPost() {
		return isset($_POST[$this->name]);
	}

	/**
	 * Resets variables when model is called.
	 *
	 * @method setVars
	 * @private
	 */
	protected function setVars() {
		$this->return_data = array();
		$this->success = true;
		$this->error_msg = '';
	}


	/**
	 * Call allows PHP to pass data into the model.
	 *
	 * @method call
	 * @public
	 * @param {Object} Data object to interact with the model.
	 * @return {Object} Return data.
	 */
	public function call($data) {
		$this->data = $data;

		$this->setVars();
		$this->main();

		return $this->returnObj();
	}

	/**
	 * Function that is called to check if it is called with Post.
	 *
	 * @method __construct
	 * @public
	 * @param notAPI {Boolean}
	 */
	public function __construct($notAPI) {
		// If not being called as API, then unset POST.
		if ($notAPI) {
			unset($_POST);
		}

		// If called as API.
		if ($this->isPost()) {
			$this->data = $this->decodePost();

			$this->setVars();
			$this->main();

			echo json_encode($this->returnObj());
		}
	}
}

?>\end{lstlisting}
}
\textbf{Test.php}

Path: /php/Test.php
{\scriptsize
\begin{lstlisting}
<!DOCTYPE html>
<html>
<head>
<title>Unit tests</title>

<link rel="stylesheet" href="../build/all.css">
</head>
<body>


<?php

/**
 * Class for testing PHP models.
 *
 * @class Test
 * @static
 */
class Test
{
	/**
	 * Method that requires all the files.
	 *
	 * @method requireAll
	 * @private
	 * @static
	 */
	private static function requireAll() {
		// Require all models.
		foreach (scandir(dirname(__DIR__) . '/models/') as $filename) {
			$path = dirname(__DIR__) . '/models/' . $filename;

			if (is_file($path)) {
				require_once($path);
			}
		}
	}

	/**
	 * Method that should be called when starting a test.
	 *
	 * @method testStart
	 * @param testName {String}
	 * @private
	 * @static
	 */
	private static function testStart($testName) {
		$startHTML = <<<HTML
		<h1>$testName</h1>

		<table class="test">
			<thead>
				<th>Test</th>
				<th>Description</th>
				<th>Input data</th>
				<th>Expected result</th>
				<th>Result</th>
				<th>Pass</th>
			</thead>
			<tbody>
HTML;

		echo $startHTML;
	}

	/**
	 * Method that should be called when finished a test.
	 *
	 * @method testEnd
	 * @private
	 * @static
	 */
	private static function testEnd() {
		$endHTML = <<<HTML
			</tbody>
		</table>
HTML;

		echo $endHTML;
	}

	/**
	 * When making a single test, call this method.
	 * Puts row in table.
	 *
	 * @method unitTest
	 * @private
	 * @static
	 */
	private static function unitTest($modelName, $description, $data, $expected) {
		$model = new $modelName(true);
		$result = json_encode($model->call($data));
		$isPass = json_encode($result == $expected);

		echo '<tr class="' . $isPass . '">';
		echo '<td>' . $modelName . '</td>';
		echo '<td>' . $description . '</td>';
		echo '<td>' . json_encode($data) . '</td>';
		echo '<td>' . $expected . '</td>';
		echo '<td>' . $result . '</td>';
		echo '<td>' . $isPass . '</td>';
		echo '</tr>';
	}

	/**
	 * Method that loads the test.
	 *
	 * @method loadTest
	 * @private
	 * @static
	 */
	private static function loadTest($file_name) {
		require_once(dirname(__DIR__) . '/tests/' . $file_name . '.php');
	}


	/**
	 * Method that is called first.
	 * Calls all tests individually.
	 *
	 * @method init
	 * @public
	 */
	public static function init() {
		self::requireAll();
		
		Database::reset();

		self::loadTest('RegisterLogin');
		self::loadTest('Leagues');
		//self::loadTest('Main.php');

		// Resets the database
		Database::reset();
	}
}

Test::init();

?>


</body>
</html>\end{lstlisting}
}
\subsubsection{Parent Classes}
\textbf{EloRating.php}

Path: /superclasses/EloRating.php
{\scriptsize
\begin{lstlisting}
<?php

require_once(dirname(__DIR__) . '/php/Database.php');


/**
 * Class with functions for calculating new elo rating.
 *
 * @class EloRating
 */
class EloRating
{
	/**
	 * New rating of the player.
	 *
	 * @property new_rating
	 * @public
	 * @type Integer
	 */
	public $new_rating;
	/**
	 * Rating change.
	 * New rating - old rating.
	 *
	 * @property rating_change
	 * @public
	 * @type Integer
	 */
	public $rating_change;
	/**
	 * K factor for the weight of rating change.
	 *
	 * @property k_factor
	 * @public
	 * @type Integer
	 */
	public $k_factor = 20;


	/**
	 * SQL query for getting the player's latest rating.
	 *
	 * @property query_get_rating
	 * @type String
	 * @private
	 * @static
	 */
	private static $query_get_rating = <<<SQL
		SELECT
		ru.rating rating
		FROM result_user_maps ru
		INNER JOIN results r
		ON r.id = ru.result_id
		WHERE
		ru.user_id = :user_id
		ORDER BY r.date DESC
		LIMIT 1
SQL;

	/**
	 * Start rating for all players.
	 *
	 * @property default_rating
	 * @type Integer
	 * @public
	 * @static
	 */
	public static $default_rating = 1300;


	/**
	 * Method gets rating of a user using @property query_get_rating.
	 *
	 * @method getPlayerRating
	 * @public
	 * @static
	 * @param user_id {Integer} Id of the user to get rating.
	 * @return {Integer} Latest rating of the user.
	 */
	public static function userRating($user_id) {
		$stmt = Database::$conn->prepare(self::$query_get_rating);

		$stmt->bindParam(':user_id', $user_id);

		if ($stmt->execute()) {
			$result = $stmt->fetchAll(PDO::FETCH_ASSOC);

			if (count($result) == 0) {
				return self::$default_rating;
			}
			else {
				return $result[0]['rating'];
			}
		}
		else {
			return false;
		}
	}

	/**
	 * Method that calculates a probability of a player winning.
	 *
	 * @method expected
	 * @public
	 * @param rating_a {Integer} Rating of player A.
	 * @param rating_b {Integer} Rating of player B.
	 * @return {Float} Odds of player A beating player B.
	 * @static
	 */
	public static function expected($rating_a, $rating_b) {
		return 1 / (1 + pow(10, ($rating_b - $rating_a) / 400));
	}

	/**
	 * Method that calculates a new rating based on the score.
	 *
	 * @method newRating
	 * @public
	 * @param rating_a {Integer} Rating of player A.
	 * @param k_factor {Integer} The k factor of the match.
	 * @param points_a {Float} Result of the match (0 = defeat, 0.5 = draw, 1 = win).
	 * @param expected_a {Float} Expected result as calculated from @method expected.
	 * @return {Integer} New rating of player A.
	 * @static
	 */
	public static function newRating($rating_a, $k_factor, $points_a, $expected_a) {
		return $rating_a + $k_factor * ($points_a - $expected_a);
	}


	/**
	 * Method that is called when an instance of the class is made.
	 *
	 * @method __construct
	 * @public
	 * @param player_a_id {Integer} Id of player A.
	 * @param player_b_id {Integer} Id of player B.
	 * @param tournament_id {Integer} Id of tournament.
	 * @param player_a_score {Integer} Score that player A achieved.
	 * @param player_b_score {Integer} Score that player B achieved.
	 */
	public function __construct($player_a_id, $player_b_id, $tournament_id, $player_a_score, $player_b_score) {
		$player_a_rating = self::userRating($player_a_id, $tournament_id);
		$player_b_rating = self::userRating($player_b_id, $tournament_id);
		$points;

		// Calculate new score
		if ($player_a_score > $player_b_score) {
			$points = 1;
		}
		else if ($player_a_score == $player_b_score) {
			$points = 0.5;
		}
		else {
			$points = 0;
		}

		$expected = self::expected($player_a_rating, $player_b_rating);
		$this->new_rating = self::newRating($player_a_rating, $this->k_factor, $points, $expected);
		$this->rating_change = $this->new_rating - $player_a_rating;
	}
}

?>\end{lstlisting}
}
\textbf{ResultMethods.php}

Path: /superclasses/ResultMethods.php
{\scriptsize
\begin{lstlisting}
<?php

require_once(dirname(__DIR__) . '/php/Database.php');


/**
 * Helpful functions for dealing with results.
 *
 * @class ResultMethods
 */
class ResultMethods
{
	/**
	 * SQL query that checks whether a result exists or not.
	 *
	 * @property query_result_exists
	 * @type String
	 * @private
	 * @static
	 */
	private static $query_result_exists = <<<SQL
		SELECT COUNT(*) result_exists
		FROM results r
		INNER JOIN result_user_maps ru1
		ON r.id = ru1.result_id
		INNER JOIN result_user_maps ru2
		ON r.id = ru2.result_id
		WHERE
		ru1.user_id <> ru2.user_id AND
		ru1.user_id = :player1_id AND
		ru2.user_id = :player2_id AND
		r.tournament_id = :tournament_id
SQL;


	/**
	 * Method that tests whether a match has already been played.
	 *
	 * @method resultExists
	 * @public
	 * @static
	 * @param player1_id {Integer} Id of player in match.
	 * @param player2_id {Integer} Id of player in match.
	 * @param tournament_id {Integer} Id of tournament the match is in.
	 * @return {Boolean} Whether that result has been entered.
	 */
	public static function resultExists($player1_id, $player2_id, $tournament_id) {
		$stmt = Database::$conn->prepare(self::$query_result_exists);

		$stmt->bindParam(':player1_id', $player1_id);
		$stmt->bindParam(':player2_id', $player2_id);
		$stmt->bindParam(':tournament_id', $tournament_id);

		$stmt->execute();

		if ($stmt->fetchAll(PDO::FETCH_ASSOC)[0]['result_exists']) {
			return true;
		}
		else {
			return false;
		}
	}
}

?>\end{lstlisting}
}
\textbf{Tournament.php}

Path: /superclasses/Tournament.php
{\scriptsize
\begin{lstlisting}
<?php

require_once(dirname(__DIR__) . '/php/Database.php');
require_once(dirname(__DIR__) . '/php/Model.php');


/**
 * Parent class for general tournament stuff.
 *
 * @class Tournament
 * @extends Model
 */
class Tournament extends Model
{
	/**
	 * SQL query string for fetching the players in the tournament.
	 *
	 * @property query_players
	 * @type String
	 * @private
	 */
	private $query_players = <<<SQL
		SELECT u.id, u.first_name, u.last_name
		FROM users u
		INNER JOIN tournament_user_maps tu
		ON tu.user_id = u.id
		WHERE tu.tournament_id = :id AND tu.is_player = true
SQL;
	/**
	 * SQL query string for checking whether the user is a player in a particular league.
	 *
	 * @property query_player_count
	 * @type String
	 * @private
	 */
	private $query_player_count = <<<SQL
		SELECT COUNT(*)
		FROM tournament_user_maps
		WHERE
		tournament_id = :tournament_id AND
		user_id = :user_id AND
		is_player = 1
SQL;
	/**
	 * SQL query string for telling whether a user exists.
	 *
	 * @property query_user_exist
	 * @type String
	 * @private
	 */
	private $query_user_exist = <<<SQL
		SELECT COUNT(*)
		FROM users
		WHERE id = :id
SQL;
	/**
	 * SQL query string for checking whether the user is league manager of a particular league.
	 *
	 * @property query_league_manager_count
	 * @type String
	 * @private
	 */
	private $query_league_manager_count = <<<SQL
		SELECT COUNT(*)
		FROM tournament_user_maps
		WHERE
		tournament_id = :tournament_id AND
		user_id = :user_id AND
		is_league_manager = 1
SQL;
	/**
	 * SQL query string for checking whether a tournament exists.
	 *
	 * @property query_tournament_count
	 * @type String
	 * @private
	 */
	private $query_tournament_count = <<<SQL
		SELECT COUNT(*)
		FROM tournaments
		WHERE id = :id
SQL;
	/**
	 * SQL query for attaching a user to a tournament.
	 *
	 * @property query_attach
	 * @type String
	 * @private
	 */
	private $query_attach = <<<SQL
		INSERT INTO tournament_user_maps (tournament_id, user_id, is_player, is_league_manager)
		VALUES (:tournament_id, :user_id, FALSE, FALSE)
SQL;


	/**
	 * Method for telling whether the current user is a player in the tournament.
	 *
	 * @method isPlayer
	 * @protected
	 * @param user_id {Integer} Id of the user to query.
	 * @param tournament_id {Integer} Id of the tournament to query.
	 * @return {Boolean} Whether the user is a player in that tournament.
	 */
	protected function isPlayer($user_id, $tournament_id) {
		$stmt = Database::$conn->prepare($this->query_player_count);

		$stmt->bindParam(':user_id', $user_id);
		$stmt->bindParam(':tournament_id', $tournament_id);

		if ($stmt->execute()) {
			if ($stmt->fetchAll(PDO::FETCH_ASSOC)[0]['COUNT(*)'] == '1') {
				return true;
			}
			else {
				return false;
			}
		}
		else {
			return false;
		}
	}

	/**
	 * Method for telling whether the current user is a league manager is the tournament.
	 *
	 * @method isLeagueManager
	 * @protected
	 * @param user_id {Integer} Id of the user to query.
	 * @param tournament_id {Integer} Id of the tournament to query.
	 * @return {Boolean} Whether the user is a league manager in that tournament.
	 */
	protected function isLeagueManager($user_id, $tournament_id) {
		$stmt = Database::$conn->prepare($this->query_league_manager_count);

		$stmt->bindParam(':user_id', $user_id);
		$stmt->bindParam(':tournament_id', $tournament_id);

		if ($stmt->execute()) {
			if ($stmt->fetchAll(PDO::FETCH_ASSOC)[0]['COUNT(*)'] == '1') {
				return true;
			}
			else {
				return false;
			}
		}
		else {
			return false;
		}
	}

	/**
	 * Attaches a user to a tournament.
	 *
	 * @method attachUser
	 * @protected
	 * @param tournament_id {Integer} Id of the tournament.
	 * @param user_id {Integer} Id of the user.
	 * @return {Boolean} Whether successful.
	 */
	protected function attachUser($tournament_id, $user_id) {
		$stmt = Database::$conn->prepare($this->query_attach);

		$stmt->bindParam(':tournament_id', $tournament_id);
		$stmt->bindParam(':user_id', $user_id);

		if ($stmt->execute()) {
			return true;
		}
		else {
			return false;
		}
	}

	/**
	 * Checks whether a tournament exists from parameter id.
	 *
	 * @method tournamentExistsId
	 * @protected
	 * @param tournament_id {Integer} Id of tournament
	 * @return {Boolean} Whether tournament exists.
	 */
	protected function tournamentExistsId($tournament_id) {
		$stmt = Database::$conn->prepare($this->query_tournament_count);

		$stmt->bindParam(':id', $tournament_id);

		if ($stmt->execute() && $stmt->fetchAll(PDO::FETCH_ASSOC)[0]['COUNT(*)'] == '1') {
			return true;
		}
		else {
			return false;
		}
	}

	/**
	 * Checks whether a tournament exists.
	 *
	 * @method tournamentExists
	 * @protected
	 * @return {Boolean} Whether tournament exists.
	 */
	protected function tournamentExists() {
		return $this->tournamentExistsId($this->data['tournament_id']);
	}

	/**
	 * Checks whether a user exists.
	 *
	 * @method userExists
	 * @protected
	 * @param user_id {Integer} Id of the user to test.
	 * @return {Boolean} Whether a user exists.
	 */
	protected function userExists($user_id) {
		$stmt = Database::$conn->prepare($this->query_user_exist);

		$stmt->bindParam(':id', $user_id);

		if ($stmt->execute() && $stmt->fetchAll(PDO::FETCH_ASSOC)[0]['COUNT(*)'] == '1') {
			return true;
		}
		else {
			return false;
		}
	}
}

?>\end{lstlisting}
}
\textbf{TournamentManagerAlter.php}

Path: /superclasses/TournamentManagerAlter.php
{\scriptsize
\begin{lstlisting}
<?php

require_once(dirname(__DIR__) . '/superclasses/Tournament.php');


session_start();


/**
 * Parent class of classes that add or remove league managers.
 *
 * @class TournamentManagerAlter
 * @extends Tournament
 */
class TournamentManagerAlter extends Tournament
{
	/**
	 * SQL query string for changing league manager status.
	 *
	 * @property query
	 * @type String
	 * @private
	 */
	private $query = <<<SQL
		UPDATE tournament_user_maps
		SET is_league_manager = :is_league_manager
		WHERE
		user_id = :user_id AND
		tournament_id = :tournament_id
SQL;

	/**
	 * SQL query string for counting the number of league managers.
	 *
	 * @property query_managers_count
	 * @type String
	 * @private
	 */
	private $query_managers_count = <<<SQL
		SELECT COUNT(*)
		FROM tournament_user_maps
		WHERE
		is_league_manager = TRUE AND
		tournament_id = :tournament_id
SQL;

	/**
	 * Database object for executing @property query.
	 *
	 * @property stmt
	 * @type Object
	 * @private
	 */
	private $stmt;


	/**
	 * Method for finding the number of league manager.
	 *
	 * @method noOfLeagueManagers
	 * @private
	 * @return {Integer} Number of league managers.
	 */
	private function noOfLeagueManagers() {
		$stmt = Database::$conn->prepare($this->query_managers_count);

		$stmt->bindParam(':tournament_id', $this->data['tournament_id']);

		if ($stmt->execute()) {
			return $stmt->fetchAll(PDO::FETCH_ASSOC)[0]['COUNT(*)'];
		}
		else {
			// 1 - so can't claim league if error.
			return 1;
		}
	}

	/**
	 * Checks any faults and executes query.
	 *
	 * @method executeQuery
	 * @private
	 */
	private function executeQuery() {
		if (!$this->stmt->execute() || $this->stmt->rowCount() != 1) {
			$this->error_msg = "Failed to execute query";
			$this->success = false;
		}
	}

	/**
	 * Verifies whether the user can carry out the task.
	 * Returns false if:
	 * - User doesn't exist.
	 * - Tournament doesn't exist.
	 * - Not a league manager.
	 *
	 * @method verifyManager
	 * @private
	 * @return {Boolean} Whether can.
	 */
	private function verifyManager() {
		$is_league_manager = $this->isLeagueManager($_SESSION['user']['id'], $this->data['tournament_id']);
		$does_tournament_exist = $this->tournamentExists();
		$does_user_exist = $this->userExists($this->data['user_id']);

		// If there are no league managers, then the league is free to be claimed.
		if ($this->noOfLeagueManagers() == 0) {
			return true;
		}

		if (!$does_tournament_exist) {
			$this->error_msg = "Tournament doesn't exist";

			return false;
		}
		else if (!$does_user_exist) {
			$this->error_msg = "User doesn't exist";

			return false;
		}
		else if ($is_league_manager) {
			return true;
		}
		else {
			$this->error_msg = "You don't have permission to do that";

			return false;
		}
	}

	/**
	 * Main method for making the user a league manager
	 *
	 * @method general
	 * @private
	 */
	private function general() {
		$this->stmt = Database::$conn->prepare($this->query);

		$this->stmt->bindParam(':user_id', $this->data['user_id']);
		$this->stmt->bindParam(':tournament_id', $this->data['tournament_id']);
		$this->stmt->bindParam(':is_league_manager', $this->is_league_manager);

		// Verify whether the user can carry out the action.
		if ($this->verifyManager()) {
			// Attach user if not already.
			$this->attachUser($this->data['tournament_id'], $this->data['user_id']);
			// Execute query
			$this->executeQuery();
		}
		else {
			$this->success = false;
		}
	}

	/**
	 * Method that checks whether the user is logged in.
	 *
	 * @method subMain
	 * @protected
	 */
	protected function subMain() {
		if (isset($_SESSION['user'])) {
			$this->general();
		}
		else {
			$this->error_msg = "You must be logged in";

			$this->success = false;
		}
	}
}

?>\end{lstlisting}
}
\textbf{TournamentPlayerAlter.php}

Path: /superclasses/TournamentPlayerAlter.php
{\scriptsize
\begin{lstlisting}
<?php

require_once(dirname(__DIR__) . '/superclasses/Tournament.php');

session_start();


/**
 * Parent class of classes that add or remove players.
 *
 * @class TournamentPlayerAlter
 * @extends Tournament
 */
class TournamentPlayerAlter extends Tournament
{
	/**
	 * SQL query string for updating a player
	 *
	 * @property query
	 * @type String
	 * @private
	 */
	private $query = <<<SQL
		UPDATE tournament_user_maps
		SET is_player = :is_player
		WHERE
		user_id = :user_id AND
		tournament_id = :tournament_id
SQL;

	/**
	 * Database object for executing @property query.
	 *
	 * @property stmt
	 * @type Object
	 * @private
	 */
	private $stmt;


	/**
	 * Checks any faults and executes query.
	 *
	 * @method executeQuery
	 * @private
	 */
	private function executeQuery() {
		if (!$this->stmt->execute() || $this->stmt->rowCount() != 1) {
			$this->error_msg = "Failed to execute query";
			$this->success = false;
		}
	}

	/**
	 * Method for verifying whether the user can carry out the task.
	 * Returns false if:
	 * - User doesn't exist.
	 * - Tournament doesn't exist.
	 * - Either:
	 *   - Not a league manager.
	 *   - Altering someone else.
	 *
	 * @method verifyPlayer
	 * @private
	 * @return {Boolean} Whether can.
	 */
	private function verifyPlayer() {
		$is_league_manager = $this->isLeagueManager($_SESSION['user']['id'], $this->data['tournament_id']);
		$does_tournament_exist = $this->tournamentExists();
		$does_user_exist = $this->userExists($this->data['user_id']);

		if (!$does_tournament_exist) {
			$this->error_msg = "Tournament doesn't exist";

			return false;
		}
		else if (!$does_user_exist) {
			$this->error_msg = "User doesn't exist";

			return false;
		}
		else if ($is_league_manager || $this->data['user_id'] == $_SESSION['user']['id']) {
			return true;
		}
		else {
			$this->error_msg = "You don't have permission to do that";

			return false;
		}
	}

	/**
	 * Method that creates database object.
	 * Binds parameters.
	 * Checks to see if able to execute query.
	 * Then execute query.
	 * 
	 * @method general
	 * @private
	 */
	private function general() {
		$this->stmt = Database::$conn->prepare($this->query);

		$this->stmt->bindParam(':user_id', $this->data['user_id']);
		$this->stmt->bindParam(':tournament_id', $this->data['tournament_id']);
		$this->stmt->bindParam(':is_player', $this->is_player);

		// Verify whether the user can carry out the action.
		if ($this->verifyPlayer()) {
			// Attach user if not already.
			$this->attachUser($this->data['tournament_id'], $this->data['user_id']);
			// Execute query
			$this->executeQuery();
		}
		else {
			$this->success = false;
		}
	}

	/**
	 * Method that checks login.
	 * Then calls @method general
	 *
	 * @method subMain
	 * @protected
	 */
	protected function subMain() {
		if (isset($_SESSION['user'])) {
			$this->general();
		}
		else {
			$this->error_msg = "You must be logged in";

			$this->success = false;
		}
	}
}

?>\end{lstlisting}
}
\textbf{Validate.php}

Path: /superclasses/Validate.php
{\scriptsize
\begin{lstlisting}
<?php

/**
 * Class used to validate input text.
 *
 * @class Validate
 */
class Validate
{
	/**
	 * Returns standard data such as whether success, error message, and correction.
	 *
	 * @method returnData
	 * @private
	 * @static
	 * @param success {Boolean} Whether a success or not.
	 * @param error_msg {String} Error message.
	 * @return {Array}
	 *   @return [].success {Boolean} Whether a success or not.
	 *   @return [].error_msg {String} Error message.
	 */
	private static function returnData($success, $error_msg) {
		return array(
			'success' => $success,
			'error_msg' => $error_msg
		);
	}


	/**
	 * Valididates user's name.
	 * Can be first or last name.
	 *
	 * @method userName
	 * @static
	 * @param name {String} Name to be checked.
	 * @param variableName {String} Name of variable e.g. 'First name' or 'Last name'.
	 * @return {Object} Return of @method returnData
	 */
	public static function userName($name, $variableName) {
		// Regex, matching name.
		preg_match("/^[A-z']+$/", $name, $match);

		// Check only alphabetical characters using regex.
		if ($match[0] != $name) {
			return self::returnData(false, $variableName . ' must only be alphabetical characters');
		}
		// Check length > 1.
		if (strlen($name) < 2) {
			return self::returnData(false, $variableName . ' must contain at least 2 characters');
		}
		// Check length <= 30.
		if (strlen($name) > 30) {
			return self::returnData(false, $variableName . ' must contain at most 30 characters');
		}

		return self::returnData(true, null);
	}

	/**
	 * Validates tournament name.
	 *
	 * @method tournamentName
	 * @static
	 * @param name {String}
	 * @return {Object} Return of @method returnData
	 */
	public static function tournamentName($name) {
		// Regex
		preg_match("/^[A-z0-9' !@#$%^&*()£\/?,.]+$/", $name, $match);

		// Check only a-zA-Z0-9 and spaces using regex.
		if ($match[0] != $name) {
			return self::returnData(false, 'League name can only contain alphanumerical or !@#$%^&*()£/?,. characters');
		}
		// Check length > 5.
		if (strlen($name) < 5) {
			return self::returnData(false, 'League name must be at least 5 characters');
		}
		// Check length <= 40.
		if (strlen($name) > 40) {
			return self::returnData(false, 'League name must be at most 40 characters');
		}

		return self::returnData(true, null);
	}

	/**
	 * Validates tournament description.
	 *
	 * @method tournamentDescription
	 * @static
	 * @param description {String}
	 * @return {Object} Return of @method returnData
	 */
	public static function tournamentDescription($description) {
		if (strlen($description) > 255) {
			return self::returnData(false, 'League description must be at most 255 characters');
		}

		return self::returnData(true, null);
	}

	/**
	 * Validates email address.
	 *
	 * @method email
	 * @static
	 * @param email {String}
	 * @return {Object} Return of @method returnData
	 */
	public static function email($email) {
		// Regex
		preg_match("/^[a-zA-Z0-9_.+-]+@[a-zA-Z0-9-]+\.[a-zA-Z0-9-.]+$/", $email, $match);

		// Check email is entered
		if (strlen($email) < 1) {
			return self::returnData(false, 'You must enter an email');
		}
		// Check using valid email regex.
		if ($match[0] != $email) {
			return self::returnData(false, 'Email is not valid');
		}

		return self::returnData(true, null);
	}

	/**
	 * Validates password.
	 *
	 * @method password
	 * @static
	 * @param password {String}
	 * @return {Object} Return of @method returnData
	 */
	public static function password($password) {
		// Check length > 5.
		if (strlen($password) < 6) {
			return self::returnData(false, 'Password must be at least 6 characters');
		}

		return self::returnData(true, null);
	}

	/**
	 * Validates phone number.
	 *
	 * @method phoneNumber
	 * @static
	 * @param phoneNumber {String}
	 * @return {Object} Return of @method returnData
	 */
	public static function phoneNumber($number) {
		// Regex
		preg_match("/^[0-9 ]+$/", $number, $match);

		// Check only numerical characters, using regex.
		if ($match[0] != $number) {
			return self::returnData(false, 'Phone number must only contain numerical characters and spaces');
		}
		// Check length < 5.
		if (strlen($number) < 5) {
			return self::returnData(false, 'Phone number must be at least 6 characters long');
		}
		// Check length > 20.
		if (strlen($number) > 20) {
			return self::returnData(false, 'Phone number must be at most 20 characters long');
		}

		return self::returnData(true, null);
	}
}

?>\end{lstlisting}
}
\newpage
\subsection{JavaScript}
\subsubsection{Configuration}
\textbf{app.js}

Path: /app/app.js
{\scriptsize
\begin{lstlisting}
/**
 * Main Angular object.
 *
 * @var app
 */
var app = angular.module('computing-project', ['ngRoute']);\end{lstlisting}
}
\textbf{config.js}

Path: /app/config.js
{\scriptsize
\begin{lstlisting}
/****************************************
 * Angular config.                      *
 * Mainly used to configure the routes. *
 ****************************************/
app.config(['$routeProvider', function ($routeProvider)
{
	$routeProvider
	.when('/', {
		templateUrl: 'views/home.html',
		controller: 'HomeCtrl'
	})
	.when('/user', {
		templateUrl: 'views/user-search-page.html',
		controller: 'UserSearchCtrl'
	})
	.when('/profile', {
		templateUrl: 'views/profile.html',
		controller: 'ProfileCtrl'
	})
	.when('/profile/edit', {
		templateUrl: 'views/profile-edit.html',
		controller: 'ProfileEditCtrl'
	})
	.when('/profile/:userId', {
		templateUrl: 'views/profile.html',
		controller: 'ProfileCtrl'
	})
	.when('/league', {
		templateUrl: 'views/league-search-page.html',
		controller: 'LeagueSearchCtrl'
	})
	.when('/league/create', {
		templateUrl: 'views/league-edit.html',
		controller: 'LeagueCreateCtrl'
	})
	.when('/league/:leagueId/edit', {
		templateUrl: 'views/league-edit.html',
		controller: 'LeagueEditCtrl'
	})
	.when('/league/:leagueId', {
		templateUrl: 'views/league.html',
		controller: 'LeagueCtrl'
	})
	.when('/result/enter/:tournamentId/:player1Id/:player2Id', {
		templateUrl: 'views/result-enter.html',
		controller: 'ResultEnterCtrl'
	})
	.when('/login', {
		templateUrl: 'views/login.html',
		controller: 'LoginCtrl'
	})
	.when('/register', {
		templateUrl: 'views/login.html',
		controller: 'LoginCtrl'
	})
	.otherwise({
		redirectTo: '/'
	});
}]);\end{lstlisting}
}
\subsubsection{Controllers}
\textbf{HomeCtrl.js}

Path: /controllers/HomeCtrl.js
{\scriptsize
\begin{lstlisting}
/**
 * Controller for the home page.
 *
 * @controller HomeCtrl
 */
app.controller('HomeCtrl', function ()
{
	
});\end{lstlisting}
}
\textbf{LeagueCreateCtrl.js}

Path: /controllers/LeagueCreateCtrl.js
{\scriptsize
\begin{lstlisting}
/**
 * Controller for creating a league.
 *
 * @controller LeagueCreateCtrl
 */
app.controller('LeagueCreateCtrl', function ($scope, $location, CallModel)
{
	/**
	 * The text that goes in the edit page.
	 *
	 * @var $scope.editOrCreate
	 * @type String
	 */
	$scope.editOrCreate = 'Create';
	/**
	 * Object where error message is and tells whether there is an error message.
	 *
	 * @property $scope.response
	 * @type Object
	 */
	$scope.response = {
		success: true
	}
	
	
	/**
	 * Event that is called when 'Create' button is clicked.
	 * Creating the league.
	 *
	 * @method $scope.eventUpdate
	 */
	$scope.eventUpdate = function () {
		CallModel.fetch('TournamentCreate', {
			name: $scope.name,
			description: $scope.description
		},
		{
			success: function (response) {
				alert('Successfully created the league');

				$location.path('/league/' + response.id);
			},
			fail: function (response) {
				$scope.response = response;
			}
		});
	};


	(function () {
		CallModel.redirectIfNotLoggedIn();
	}());
});\end{lstlisting}
}
\textbf{LeagueCtrl.js}

Path: /controllers/LeagueCtrl.js
{\scriptsize
\begin{lstlisting}
/**
 * League page controller.
 *
 * @controller LeagueCtrl
 */
app.controller('LeagueCtrl', function ($scope, $http, $location, $routeParams, CallModel)
{
	/**
	 * String that determines which sub page the user is on.
	 *
	 * @var $scope.subPage
	 * type String
	 */
	$scope.subPage = 'table';
	/**
	 * Object that contains the table data.
	 *
	 * @var $scope.table
	 * type Object
	 */
	$scope.table;


	/**
	 * Method for getting league data from @class TournamentGet.
	 *
	 * @method getLeague
	 */
	function getLeague() {
		CallModel.fetch('TournamentGet', {
			id: $routeParams.leagueId
		},
		{
			success: function (response) {
				$scope.data = response;
			},
			fail: function (response) {
				alert(response.error_msg);

				$location.path('/');
			}
		});

		CallModel.fetch('TournamentLeagueTable', {
			id: $routeParams.leagueId
		},
		{
			success: function (response) {
				$scope.table = response.table;
			}
		});
	}


	(function () {
		CallModel.redirectIfNotLoggedIn();
		getLeague();
	}());
});\end{lstlisting}
}
\textbf{LeagueEditCtrl.js}

Path: /controllers/LeagueEditCtrl.js
{\scriptsize
\begin{lstlisting}
/**
 * Controller for editing a league.
 *
 * @controller LeagueEditCtrl
 */
app.controller('LeagueEditCtrl', function ($scope, $window, $location, $routeParams, CallModel)
{
	/**
	 * The text that goes in the edit page.
	 *
	 * @var $scope.editOrCreate
	 * @type String
	 */
	$scope.editOrCreate = 'Edit';
	/**
	 * Id of the user.
	 * Used to prevent the user from removing themselves as a league manager.
	 *
	 * @var $scope.yourId
	 * @type Integer
	 */
	$scope.yourId = $window.sessionStorage.yourId;
	/**
	 * Variable that is used to tell the view whether the adding manager dialog is open.
	 *
	 * @var $scope.addingManager
	 * @type Boolean
	 */
	$scope.addingManager = false;
	/**
	 * Variable that is used to tell the view whether the adding player dialog is open.
	 *
	 * @var $scope.addingPlayer
	 * @type Boolean
	 */
	$scope.addingPlayer = false;
	/**
	 * Object where error message is and tells whether there is an error message.
	 *
	 * @property $scope.response
	 * @type Object
	 */
	$scope.response = {
		success: true
	}

	/**
	 * Method that redirects to home page and prints error message.
	 *
	 * @method redirect
	 */
	function redirect(response) {
		alert(response.error_msg);

		$location.path('/');
	}

	/**
	 * Method for getting league data from @class TournamentGet.
	 *
	 * @method getLeague
	 */
	function getLeague() {
		CallModel.fetch('TournamentGet', {
			id: $routeParams.leagueId
		},
		{
			success: function (response) {
				$scope.data = response;
				// Fill fields with current league info.
				$scope.name = response.name;
				$scope.description = response.description;

				// Check to see if user isn't a league manager, and then redirect
				if (!response.is_league_manager) {
					response.error_msg = "You must be a league manager to edit the league";

					redirect(response);
				}
			},
			fail: redirect
		});
	}


	/**
	 * Function that is called when the update button is pressed.
	 *
	 * @method $scope.eventUpdate
	 */
	$scope.eventUpdate = function () {
		// Calls @class TournamentUpdate with info from name and description fields.
		CallModel.fetch('TournamentUpdate', {
			id: $routeParams.leagueId,
			name: $scope.name,
			description: $scope.description
		},
		{
			success: function (response) {
				$location.path('/league/' + $routeParams.leagueId);
			},
			fail: function (response) {
				$scope.response = response;
			}
		});
	}

	/**
	 * Function that is called when a player is removed from the league.
	 *
	 * @method $scope.eventRemoveUser
	 * @param userId {Integer} Id of the player to remove.
	 */
	$scope.eventRemovePlayer = function (userId) {
		CallModel.fetch('TournamentPlayerRemove', {
			user_id: userId,
			tournament_id: $routeParams.leagueId
		},
		{
			success: function (response) {
				getLeague();
			},
			fail: function (response) {
				alert(response.error_msg);
			}
		});
	};

	/**
	 * Function that is called when a league manager is removed.
	 *
	 * @method $scope.eventRemoveManager
	 * @param userId {Integer} Id of the league manager to remove.
	 */
	$scope.eventRemoveManager = function (userId) {
		CallModel.fetch('TournamentManagerRemove', {
			user_id: userId,
			tournament_id: $routeParams.leagueId
		},
		{
			success: function (response) {
				getLeague();
			}
		});
	};

	/**
	 * Function that is called by @directive cpUserSearch when adding a player.
	 *
	 * @method $scope.eventAddSpecificPlayer
	 * @param userId {Integer} Id of the player to add.
	 */
	$scope.eventAddSpecificPlayer = function (userId) {
		CallModel.fetch('TournamentPlayerAttach', {
			user_id: userId,
			tournament_id: $routeParams.leagueId
		},
		{
			success: function (response) {
				$scope.eventCancelAdding();
				getLeague();
			},
			fail: function () {
				alert("User already a player");
			}
		});
	};

	/**
	 * Function that is called by @directive cpUserSearch when adding a league manager.
	 *
	 * @method $scope.eventAddSpecificManager
	 * @param userId {Integer} Id of the player to add.
	 */
	$scope.eventAddSpecificManager = function (userId) {
		CallModel.fetch('TournamentManagerAttach', {
			user_id: userId,
			tournament_id: $routeParams.leagueId
		},
		{
			success: function (response) {
				$scope.eventCancelAdding();
				getLeague();
			},
			fail: function () {
				alert("User already a league manager");
			}
		})
	}

	/**
	 * Function that opens adding league manager dialog but closes others.
	 *
	 * @method $scope.eventAddManager
	 */
	$scope.eventAddManager = function () {
		$scope.addingManager = true;
		$scope.addingPlayer = false;
	};

	/**
	 * Function that opens adding player dialog but closes others.
	 *
	 * @method $scope.eventAddPlayer
	 */
	$scope.eventAddPlayer = function () {
		$scope.addingManager = false;
		$scope.addingPlayer = true;
	};

	/**
	 * Function that closes all adding user dialogs.
	 *
	 * @method $scope.eventCancelAdding
	 */
	$scope.eventCancelAdding = function () {
		$scope.addingManager = false;
		$scope.addingPlayer = false;
	};

	/**
	 * Function that calls @class TournamentDelete, to delete the league.
	 *
	 * @method $scope.eventDelete
	 */
	 $scope.eventDelete = function () {
	 	CallModel.fetch('TournamentDelete', {
	 		id: $routeParams.leagueId
	 	}, {
	 		success: function (response) {
	 			alert('Successfully deleted the league');

	 			$location.path('/profile');
	 		}
	 	})
	 };



	(function () {
		/* 
		 * Checks to see if logged in.
		 * If not, redirects to login page.
		 */
		CallModel.redirectIfNotLoggedIn();

		getLeague();
	}());
});\end{lstlisting}
}
\textbf{LeagueSearchCtrl.js}

Path: /controllers/LeagueSearchCtrl.js
{\scriptsize
\begin{lstlisting}
/**
 * Controller used to search for a leauge.
 *
 * @controller UserSearchCtrl
 */
app.controller('LeagueSearchCtrl', function ($scope, $location)
{
	/**
	 * Method that is called when a league is clicked.
	 * Clicked from @directive cpLeagueSearch.
	 *
	 * @method $scope.eventClickLeague
	 * @param leagueId
	 */
	$scope.eventClickLeague = function (leagueId) {
		$location.path('/league/' + leagueId);
	}
});\end{lstlisting}
}
\textbf{LoginCtrl.js}

Path: /controllers/LoginCtrl.js
{\scriptsize
\begin{lstlisting}
/**
 * Controller for managing user login.
 * Also manages user registering.
 *
 * @controller LoginCtrl
 */
app.controller('LoginCtrl', function ($scope, $window, $http, $location, $route, CallModel)
{
	$scope.response = {
		success: true
	};

	$scope.responseRegister = {
		success: true
	};

	/*
	 * If user is logged in, redirects to home page.
	 */
	 CallModel.ifLoggedIn(function () {
	 	$location.path('/');
	 },
	 function () {});


	 /**
	  * Method for getting user data and storing it in a session.
	  *
	  * @method getUserData
	  */
	 function getUserData() {
	 	CallModel.fetch('Status', {},
	 	{
	 		success: function (response) {
	 			$window.sessionStorage.yourId = response.user.id;
	 		}
	 	});
	 }


	/**
	 * Method that logs the user in.
	 *
	 * @method eventLogin
	 */
	$scope.eventLogin = function () {
		CallModel.fetch('Login', {
			email: $scope.inputEmailLogin,
			password: $scope.inputPasswordLogin
		},
		{
			success: function (response) {
				getUserData();

				alert('You have successfully logged in');
				
				$location.path('/');
			},
			fail: function (response) {
				$scope.response = response;
			}
		});
	};

	/**
	 * Method that registers the user.
	 *
	 * @method eventRegister
	 */
	$scope.eventRegister = function () {
		if ($scope.inputPasswordRegister === $scope.inputPasswordRepeatRegister) {
			CallModel.fetch('Register', {
				first_name: $scope.inputFirstName,
				last_name: $scope.inputLastName,
				email: $scope.inputEmailRegister,
				password: $scope.inputPasswordRegister
			},
			{
				success: function (response) {
					alert('You have successfully registered. You must now login.');
					
					$route.reload();
				},
				fail: function (response) {
					$scope.responseRegister = response;
				}
			});
		}
		else {
			$scope.responseRegister = {
				success: false,
				error_msg: 'Passwords do not match'
			}
		}
	};
});\end{lstlisting}
}
\textbf{NavCtrl.js}

Path: /controllers/NavCtrl.js
{\scriptsize
\begin{lstlisting}
/**
 * Controller for managing the navbar.
 *
 * @controller NavCtrl
 */
app.controller('NavCtrl', function ($scope, $window, $location, CallModel)
{
	/**
	 * Changes login boolean depending whether logged in or not.
	 *
	 * @method loginButtonText
	 */
	function loginButtonCheck() {
		CallModel.ifLoggedIn(function () {
			$scope.loggedIn = true;
		},
		function () {
			$scope.loggedIn = false;
		});
	}


	/**
	 * Event when logout button is clicked.
	 * Calls model 'Logout' then redirects to login page.
	 *
	 * @method $scope.btnLogout
	 */
	$scope.btnLogout = function () {
		CallModel.fetch('Logout', {}, {
			success: function () {
				$location.path('/login');
			}
		});
	};


	$scope.$on('$routeChangeSuccess', loginButtonCheck);
});\end{lstlisting}
}
\textbf{ProfileCtrl.js}

Path: /controllers/ProfileCtrl.js
{\scriptsize
\begin{lstlisting}
/**
 * Profile page controller.
 *
 * @controller ProfileCtrl
 */
app.controller('ProfileCtrl', function ($scope, $http, $location, $routeParams, $window, CallModel, RatingChart)
{
	/**
	 * Id of the user.
	 *
	 * @var userId
	 * @type Integer
	 */
	var userId;

	/**
	 * Variable for storing the name of the current subpage.
	 *
	 * @var $scope.currentSubPage
	 * @type String
	 */
	$scope.subPage = 'results';
	/**
	 * Is user.
	 *
	 * @var $scope.isUser
	 * @type Boolean
	 */
	 $scope.isUser = false;


	/**
	 * Method that gets user data from UserGet model.
	 *
	 * @method getUser
	 * @param user_id {Integer} Id of user to get.
	 */
	function getUser(user_id) {
		CallModel.fetch('UserGet', {
			id: user_id
		},
		{
			success: function (response) {
				$scope.data = response;
				$scope.data.full_name = response.first_name + ' ' + response.last_name;
			},
			fail: function (response) {
				alert(response.error_msg);

				$location.path('/');
			}
		});
	}

	/**
	 * Method the gets the id of the current user.
	 * Then calls @method getUser with id of current user.
	 *
	 * @method getStatus
	 * @param callback {Function} Executes after response from server.
	 */
	function getStatus(callback) {
		CallModel.fetch('Status', {}, {
			success: function (response) {
				userId = response.user.id;

				getUser(userId);
				callback();
			}
		});
	}

	/**
	 * Method that gets the users ratings for the graph.
	 *
	 * @method getRatings
	 */
	function getRatings() {
		CallModel.fetch('UserRatings', {
			user_id: userId
		}, {
			success: function (response) {
				RatingChart.inputRatings(response.ratings);
				RatingChart.draw();
			}
		});
	}


	/*
	 * Redirects if not logged in.
	 * Checks whether url user id is set.
	 * If not, calls @method getStatus.
	 * Else, calls @method getUser with id of user.
	 */
	(function () {
		CallModel.redirectIfNotLoggedIn();

		RatingChart.init();
		
		if ($routeParams.userId === undefined) {
			$scope.isUser = true;

			getStatus(function () {
				getRatings();
			});
		}
		else {
			userId = $routeParams.userId;
			$scope.isUser = $window.sessionStorage.yourId === userId;

			getUser(userId);
			getRatings();
		}
	}());
});\end{lstlisting}
}
\textbf{ProfileEditCtrl.js}

Path: /controllers/ProfileEditCtrl.js
{\scriptsize
\begin{lstlisting}
/**
 * Controller for editing a user.
 *
 * @controller
 */
app.controller('ProfileEditCtrl', function ($scope, $window, $location, CallModel)
{
	/**
	 * The user id.
	 *
	 * @property $scope.yourId
	 * @type Integer
	 */
	$scope.yourId = $window.sessionStorage.yourId;
	/**
	 * Error message to be displayed if there is an error.
	 *
	 * @property $scope.errorMsg
	 * @type String
	 */
	$scope.errorMsg;


	/**
	 * Method that is called when submit button is pressed.
	 *
	 * @method $scope.eventSubmit
	 */
	$scope.eventSubmit = function () {
		CallModel.fetch('UserUpdate', {
			first_name: $scope.first_name,
			last_name: $scope.last_name,
			home_phone: $scope.home_phone,
			mobile_phone: $scope.mobile_phone
		},
		{
			success: function (response) {
				alert("Successfully updated your profile");

				$location.path('/profile');
			},
			fail: function (response) {
				$scope.errorMsg = response.error_msg;
			}
		})
	};


	/**
	 * Method the gets the data of the user.
	 *
	 * @method getUserData
	 */
	function getUserData() {
		CallModel.fetch('UserGet', {
			id: $scope.yourId
		},
		{
			success: function (response) {
				$scope.first_name = response.first_name;
				$scope.last_name = response.last_name;
			},
			fail: function (response) {
				alert(response.error_msg);

				$location.path('/profile');
			}
		});
	}


	/*
	 * Redirects if not logged in.
	 * Then gets the user data to fill in the fields.
	 */
	 (function () {
	 	CallModel.redirectIfNotLoggedIn();
	 	getUserData();
	 }());
});\end{lstlisting}
}
\textbf{ResultEnterCtrl.js}

Path: /controllers/ResultEnterCtrl.js
{\scriptsize
\begin{lstlisting}
/**
 * Controller that allows a user to input a result.
 *
 * @controller ResultEnterCtrl
 */
app.controller('ResultEnterCtrl', function ($scope, $routeParams, $location, CallModel, DateFormat)
{
	/**
	 * The id of the tournament as defined by the route parametre.
	 *
	 * @var $scope.tournamentId
	 * @type Integer
	 */
	$scope.tournamentId = $routeParams.tournamentId;
	/**
	 * Name of the tournament.
	 *
	 * @var $scope.tournamentName
	 * @type String
	 */
	$scope.tournamentName;
	/**
	 * Player 1 data object.
	 *
	 * @var $scope.player1
	 * @type Integer
	 */
	$scope.player1 = {};
	/**
	 * Player 2 data object.
	 *
	 * @var $scope.player2
	 * @type Integer
	 */
	$scope.player2 = {};
	/**
	 * The date that the result will be entered.
	 *
	 * @var $scope.date
	 * @type String
	 */
	$scope.date = DateFormat.getString(Date.now());
	/**
	 * Error message, if there is one.
	 *
	 * @var $scope.errorMsg
	 * @type String
	 */
	$scope.errorMsg = '';


	/**
	 * Method that is called when submit button is pressed.
	 * Enters a result.
	 *
	 * @method $scope.eventSubmitResult
	 */
	$scope.eventSubmitResult = function () {
		CallModel.fetch('ResultEnter', {
			tournament_id: $routeParams.tournamentId,
			player1_id: $routeParams.player1Id,
			player2_id: $routeParams.player2Id,
			player1_score: $scope.player1.score,
			player2_score: $scope.player2.score
		},
		{
			success: function (response) {
				alert('Successfully entered result');

				$location.path('/league/' + $routeParams.tournamentId);
			},
			fail: function (response) {
				$scope.errorMsg = response.error_msg;
			}
		})
	};


	/**
	 * Method for getting information about the tournament.
	 *
	 * @method getTournamentData
	 */
	function getTournamentData() {
		CallModel.fetch('TournamentGet', {
			id: $routeParams.tournamentId
		},
		{
			success: function (response) {
				$scope.tournamentName = response.name;
			}
		});
	}

	/**
	 * Method that gets the player data.
	 *
	 * @method getPlayerData
	 */
	function getPlayerData() {
		CallModel.fetch('UserGet', {
			id: $routeParams.player1Id
		},
		{
			success: function (response) {
				$scope.player1.fullName = response.first_name + ' ' + response.last_name;
				$scope.player1.id = response.id;
			}
		});

		CallModel.fetch('UserGet', {
			id: $routeParams.player2Id
		},
		{
			success: function (response) {
				$scope.player2.fullName = response.first_name + ' ' + response.last_name;
				$scope.player2.id = response.id;
			}
		})
	}


	/*
	 * Redirects if not logged in.
	 * Gets data about tournament and user.
	 */
	(function () {
		CallModel.redirectIfNotLoggedIn();
		getTournamentData();
		getPlayerData();
	}());
});\end{lstlisting}
}
\textbf{UserSearchCtrl.js}

Path: /controllers/UserSearchCtrl.js
{\scriptsize
\begin{lstlisting}
/**
 * Controller used to search for a user.
 *
 * @controller UserSearchCtrl
 */
app.controller('UserSearchCtrl', function ($scope, $location)
{
	/**
	 * Method that is called when a user is clicked.
	 * Clicked from @directive cpUserSearch.
	 *
	 * @method $scope.eventClickUser
	 * @param userId
	 */
	$scope.eventClickUser = function (userId) {
		$location.path('/profile/' + userId);
	}
});\end{lstlisting}
}
\subsubsection{Directives}
\textbf{cpFixtures.js}

Path: /directives/cpFixtures.js
{\scriptsize
\begin{lstlisting}
/**
 * Directive for displaying fixtures.
 *
 * @directive cpFixtures
 */
app.directive('cpFixtures', function ($window, CallModel)
{
	/**
	 * Varaibe for storing scope.
	 *
	 * @var self
	 */
	var self;


	/**
	 * Method that gets data about fixtures from @class FixturesGet.
	 *
	 * @method getFixtures
	 */
	function getFixtures() {
		CallModel.fetch('FixturesGet', {
			player_id: self.playerId,
			tournament_id: self.tournamentId
		},
		{
			success: function (response) {
				self.fixtures = response.fixtures;
			}
		});
	}


	return {
		templateUrl: 'views/fixtures.html',
		scope: {
			playerId: '=',
			tournamentId: '=',
			isReady: '='
		},
		link: function ($scope) {
			self = $scope;

			$scope.yourId = $window.sessionStorage.yourId;

			// When loaded call @method getFixtures
			var waiting = $scope.$watch('isReady', function (success) {
				if (success === true) {
					waiting();
					getFixtures();
				}
			});
		}
	}
});\end{lstlisting}
}
\textbf{cpLeagueSearch.js}

Path: /directives/cpLeagueSearch.js
{\scriptsize
\begin{lstlisting}
/**
 * Directive used to search for a tournament.
 *
 * @directive cpLeagueSearch
 */
app.directive('cpLeagueSearch', function (CallModel)
{
	return {
		templateUrl: 'views/search.html',
		scope: {
			eventClickName: '=',
			theName: '@'
		},
		link: function (scope) {
			/**
			 * Must be empty string for search results box to fully hide
			 *
			 * @property scope.inputName
			 * @type String
			 */
			scope.inputName = '';
			

			/**
			 * Method that is called when text in input field is changed.
			 * Updates search for tournament when input is changed.
			 *
			 * @method scope.eventInputChange
			 */
			scope.eventInputChange = function () {
				CallModel.fetch('TournamentSearch', {
					name: scope.inputName
				},
				{
					success: function (response) {
						scope.results = response.tournaments;
					}
				});
			};
		}
	}
});\end{lstlisting}
}
\textbf{cpResult.js}

Path: /directives/cpResult.js
{\scriptsize
\begin{lstlisting}
/**
 * Directive for handling results.
 *
 * @directive cpResult
 */
app.directive('cpResult', function ($route, CallModel, DateFormat)
{
	/**
	 * Variable for storing scope.
	 *
	 * @var self
	 */
	var self;


	/**
	 * Function that sets CSS class for colouring result.
	 * Also calls @method getDateString for each result.
	 *
	 * @method sortResults
	 */
	function sortResults() {
		self.results.forEach(function (result) {
			if (result.score1 > result.score2) {
				result.winOrLoss = 'win';
			}
			else {
				result.winOrLoss = 'loss';
			}

			result.date = DateFormat.getString(result.date);
		});
	}

	/**
	 * Function that gets the results data from @class ResultGet.
	 *
	 * @method getResults
	 */
	function getResults() {
		CallModel.fetch('ResultGet', {
			result_id: self.resultId,
			tournament_id: self.tournamentId,
			player1_id: self.playerOneId,
			player2_id: self.playerTwoId
		},
		{
			success: function (response) {
				self.results = response.results;

				sortResults();
			}
		});
	}


	return {
		templateUrl: 'views/results.html',
		scope: {
			resultId: '=',
			tournamentId: '=',
			playerOneId: '=',
			playerTwoId: '=',
			isReady: '='
		},
		link: function ($scope) {
			self = $scope;


			/**
			 * Function that deletes a result when delete button is clicked.
			 *
			 * @method $scope.eventDelete
			 * @param id {Integer} Id of the result to be deleted.
			 */
			$scope.eventDelete = function (id) {
				CallModel.fetch('ResultDelete', {
					id: id
				},
				{
					success: function (response) {
						alert("Successfully deleted the result");

						$route.reload();
					},
					fail: function (response) {
						alert(response.error_msg);
					}
				});
			};


			// When loaded call @method getResults.
			var waiting = $scope.$watch('isReady', function (success) {
				if (success === true) {
					waiting();
					getResults();
				}
			});
		}
	}
});\end{lstlisting}
}
\textbf{cpUserSearch.js}

Path: /directives/cpUserSearch.js
{\scriptsize
\begin{lstlisting}
/**
 * Directive used to search for a user.
 *
 * @directive cpUserSearch
 */
app.directive('cpUserSearch', function (CallModel)
{
	return {
		templateUrl: 'views/search.html',
		scope: {
			eventClickName: '=',
			theName: '@'
		},
		link: function (scope, element, attrs) {
			/**
			 * Search name
			 *
			 * @property scope.inputName
			 * @type String
			 */
			scope.inputName = '';

			/**
			 * Method that is called when text in input field is changed.
			 * Updates search for user when input is changed.
			 *
			 * @method scope.eventInputChange
			 */
			scope.eventInputChange = function () {
				CallModel.fetch('UserSearch', {
					name: scope.inputName
				},
				{
					success: function (response) {
						scope.results = response.users;
					}
				});
			};
		}
	}
});\end{lstlisting}
}
\subsubsection{Services}
\textbf{CallModel.js}

Path: /services/CallModel.js
{\scriptsize
\begin{lstlisting}
/**
 * Factory for calling the PHP models.
 *
 * @factory CallModel
 */
app.factory('CallModel', function ($http, $location)
{
	return {
		/**
		 * Method for fetching a result from a model.
		 * Sends data to the model.
		 *
		 * @method fetch
		 * @param modelName {String} Name of the model to call.
		 * @param data {Object} Data to pass to the model.
		 * @param callbacks {Object} Collection of functions that will be called depending on the result.
		 *   @param success {Function} Function that is called on success.
		 *   @param fail {Function} Function that is called on fail.
		 *   @param normal {Function} Function that is called regardless.
		 */
		fetch: function (modelName, data, callbacks) {
			$http({
				url: 'php/API.php',
				method: 'POST',
				data: 'JSON=' + encodeURIComponent(JSON.stringify(data)) + '&model=' + modelName,
				headers: {
					'Content-Type': 'application/x-www-form-urlencoded'
				}
			})
			.then(function (response) {
				if (response.data.success && typeof callbacks.success == 'function') {
					callbacks.success(response.data);
				}
				else if (!response.data.success && typeof callbacks.fail == 'function') {
					callbacks.fail(response.data);
				}

				if (typeof callbacks.normal == 'function') {
					callbacks.normal(response.data);
				}
			});
		},
		/**
		 * Method that accepts two functions.
		 * One is executed if the user is logged in.
		 * The other is called if the user is not logged in.
		 *
		 * @method ifLoggedIn
		 * @param callbackTrue {Function} Called if logged in.
		 * @param callbackFalse {Function} Called if not logged in.
		 */
		ifLoggedIn: function (callbackTrue, callbackFalse) {
			this.fetch('Status', {}, {
				success: function (response) {
					if (response.logged_in) {
						callbackTrue();
					}
					else {
						callbackFalse();
					}
				}
			});
		},
		/**
		 * Method that redirects to login page if not logged in.
		 *
		 * @method redirectIfNotLoggedIn
		 */
		redirectIfNotLoggedIn: function () {
			this.ifLoggedIn(function () {}, function () {
				$location.path('/login');
			});
		}
	};
});\end{lstlisting}
}
\textbf{DateFormat.js}

Path: /services/DateFormat.js
{\scriptsize
\begin{lstlisting}
/**
 * Factory that returns the site's format of date.
 *
 * @factory DateFormat
 */
app.factory('DateFormat', function ()
{
	/**
	 * Days of the week for date format.
	 *
	 * @var days
	 */
	var days = ['SUN', 'MON', 'TUE', 'WED', 'THU', 'FRI', 'SAT'];
	/**
	 * Months of the year for date format.
	 *
	 * @var months
	 */
	var months = ['JAN', 'FEB', 'MAR', 'APR', 'MAY', 'JUN', 'JUL', 'AUG', 'SEP', 'OCT', 'NOV', 'DEC'];


	return {
		/**
		 * Function that returns formated output date from input date.
		 *
		 * @method getDateString
		 * @param dateString {String} Input date string.
		 * @return {String} Output formated date string.
		 */
		getString: function (dateString) {
			var date = new Date(dateString);

			return days[date.getDay()] + ' ' + date.getDate() + ' ' + months[date.getMonth()] + ' ' + date.getFullYear();
		}
	}
});\end{lstlisting}
}
\textbf{RatingChart.js}

Path: /services/RatingChart.js
{\scriptsize
\begin{lstlisting}
/**
 * Factory for handling the rating graph.
 *
 * @factory RatingChart
 */
app.factory('RatingChart', function ()
{
	/**	 *
	 * @var ctx
	 * @private
	 */
	var ctx;
	/**
	 * Chart object for talking to Chart.JS library.
	 *
	 * @var chart
	 * @private
	 */
	var chart;
	/**
	 * List of dates corresponding to a rating.
	 *
	 * @var dates
	 * @private
	 */
	var dates = [];
	/**
	 * List of ratings for the graph.
	 *
	 * @var averageRating
	 * @private
	 */
	var averageRating = [];
	/**
	 * Data object for drawing the chart.
	 *
	 * @var data
	 * @private
	 */
	var data = {
		labels: dates,
		datasets: [
			{
				label: "Average rating",
				fillColor: "rgba(220,220,220,0.2)",
				strokeColor: "rgba(220,220,220,1)",
				pointColor: "rgba(220,220,220,1)",
				pointStrokeColor: "#fff",
				pointHighlightFill: "#fff",
				pointHighlightStroke: "rgba(220,220,220,1)",
				data: averageRating
			}
		]
	};
	/**
	 * Configuration for drawing the chart.
	 *
	 * @var options
	 * @private
	 */
	var options = {
		scaleShowGridLines : true,
		scaleGridLineColor : "rgba(0,0,0,.05)",
		scaleGridLineWidth : 1,
		scaleShowHorizontalLines: true,
		scaleShowVerticalLines: true,
		bezierCurve : true,
		bezierCurveTension : 0.4,
		pointDot : true,
		pointDotRadius : 4,
		pointDotStrokeWidth : 1,
		pointHitDetectionRadius : 20,
		datasetStroke : true,
		datasetStrokeWidth : 2,
		datasetFill : true,
		legendTemplate : ""
	};


	return {
		/**
		 * Method that filters the ratings into different tournaments.
		 *
		 * @method inputRatings
		 * @param ratings {Array} Array of all the ratings over time.
		 */
		inputRatings: function (ratings) {
			// Sort rating in reverse order.
			ratings.sort(function (a, b) {
				return a.date > b.date;
			});

			// Calculate average rating.
			ratings.forEach(function (rating) {
				averageRating.push(parseInt(rating.rating));
				dates.push(rating.date);
			});
		},
		/**
		 * Call this method when initialising the graph.
		 *
		 * @method init
		 */
		init: function () {
			ctx = document.getElementById('canvas-rating-chart').getContext('2d');
			chart = new Chart(ctx);

			// Reset arrays without deleting reference
			dates.splice(0, dates.length);
			averageRating.splice(0, averageRating.length);

			dates.push('');
			averageRating.push(1300);
		},
		/**
		 * This method is called when drawing the graph.
		 *
		 * @method draw
		 */
		draw: function () {
			chart.Line(data, options);
		}
	}
});\end{lstlisting}
}
\newpage
\subsection{CSS}
\textbf{container.css}

Path: /css/container.css
{\scriptsize
\begin{lstlisting}
.box-container {
	padding: 2rem 2rem 0rem 2rem;
	margin-bottom: 4rem;

	border-radius: 1rem;

	background-color: #FFF;
}

.box-container.side {
	padding-bottom: 0;

	border-width: 0.3rem;
	border-color: #DADADA;
	border-style: solid;

	background-color: #E9E9E9;
}

.box-container.side h2 {
	color: #999;
	text-transform: uppercase;
}

.box-container.side span {
	line-height: auto;
}

.box-container .buffer {
	height: 1.5rem;
}


.error-container {
	background-color: #FFD4CC;
	color: #DD381D;

	border-radius: 1rem;

	padding: 0.5rem 1.5rem 0.5rem 1.5rem;

	margin-bottom: 1rem;
}

.error-container ul {
	margin: 0;

	list-style: none;
}

.error-container li {
	margin: 0;
}\end{lstlisting}
}
\textbf{league-table.css}

Path: /css/league-table.css
{\scriptsize
\begin{lstlisting}
.league-table {
	margin-bottom: 2rem;

	table-layout: fixed;
}\end{lstlisting}
}
\textbf{list.css}

Path: /css/list.css
{\scriptsize
\begin{lstlisting}
.list-side {
	margin-bottom: 2rem;
}

.list-side li {
	margin: 0;
	margin-bottom: 0.5rem;

	list-style: none;
	line-height: normal;
}

.list-side a {
	text-decoration: none;
}\end{lstlisting}
}
\textbf{misc.css}

Path: /css/misc.css
{\scriptsize
\begin{lstlisting}
.create-league {
	margin-bottom: 2rem;
}

.two-rem {
	margin-bottom: 2rem;
}

.one-rem {
	margin-bottom: 1rem;
}

.half-rem {
	margin-bottom: 0.5rem;
}\end{lstlisting}
}
\textbf{navbar.css}

Path: /css/navbar.css
{\scriptsize
\begin{lstlisting}
.nav {
	list-style: none;

	margin-bottom: 2rem;
	overflow: hidden;
}

.nav li {
	display: inline-block;
	float: left;
}

.nav.sub a {
	color: #999;

	padding: 1rem;

	border-style: solid;
	border-width: 0 0 0.3rem 0;
	border-color: #E9E9E9;
}

.nav.sub a.active {
	color: #1EAEDB;

	border-color: #1EAEDB;
}


#div-navbar {
	background-color: #2AA8E5;
	height: 6rem;
	margin-bottom: 4rem;
}

#div-navbar ul {
	padding: 0;
	margin: 0;
	list-style: none;
}

#div-navbar li {
	display: inline;
	float: right;
}

#div-navbar a {
	padding: 1.5rem;
	margin: 0;
	color: white;
	text-decoration: none;
	display: block;
	font-size: 2rem;
}

#div-navbar li a:hover {
	background-color: #1493D1;
	color: white;
}

#div-navbar #a-title {
	float: left;
}

@media (max-width: 750px) {
	#div-navbar {
		height: auto;
		overflow: auto;
	}

	#div-navbar ul {
		display: inline-block;
		width: 100%;
		transform: rotate(180deg);
	}

	#div-navbar li {
		float: none;
	}

	#div-navbar li a {
		transform: rotate(180deg);
	}

	#div-navbar #a-title {
		width: auto;
	}
}\end{lstlisting}
}
\textbf{result.css}

Path: /css/result.css
{\scriptsize
\begin{lstlisting}
.result {
	margin-bottom: 2rem;
}

.result .head {
	padding: 0.5rem;

	color: #999;
}

.result .body {
	padding: 0.5rem;
}

.result td {
	padding: 0;
}

.result .centre-align {
	text-align: center;
}

.result .player1-name {
	float: right;
	text-align: right;
}

.result .player2-name {
	float: left;
	text-align: left;
}

.result .score {
	color: white;
	padding: 0.5rem;
	border-radius: 0.3rem;
}

.result .score.win {
	background-color: #63C016;
}

.result .score.loss {
	background-color: #D81D18;
}\end{lstlisting}
}
\textbf{tags.css}

Path: /css/tags.css
{\scriptsize
\begin{lstlisting}
body {
	background-color: #E9E9E9;
	font-family: 'Open Sans', sans-serif;
	line-height: 3rem;
	font-color: #3C3C3C;
}


h1 {
	font-size: 3rem;
}


h2 {
	font-size: 2rem;
	margin-bottom: 1rem;
}


input {
	margin-bottom: 2rem;

	border-radius: 0.5rem !important;
}


td {
	border-style: none;
}


button {
	margin-bottom: 2rem;

	text-transform: none;
	font-size: 1.5rem;

	background-color: #1CAFF6;
	color: #FFF;

	border-width: 0;
	border-radius: 0.5rem;
}

button:hover {
	background-color: #3BBAF7;
	color: #FFF;
}

a:link {
	text-decoration: none;
}


hr {
	margin: 2rem 0 2rem;
}\end{lstlisting}
}
\textbf{test.css}

Path: /css/test.css
{\scriptsize
\begin{lstlisting}
.test {
	margin: 4rem;
}

.test tr.true {
	color: green;
}

.test tr.false {
	color: red;
}\end{lstlisting}
}
\newpage
\subsection{HTML}
\textbf{fixtures.html}

Path: /views/fixtures.html
{\scriptsize
\begin{lstlisting}
<!-- View for @directive cpFixtures -->
<div>
	<!-- Repeats for every fixture -->
	<div class="result" ng-repeat="n in fixtures">
		<div class="head">
			<a ng-href="#/league/{{ n.tournament_id }}" ng-bind="n.tournament_name"></a>
			<a class="u-pull-right" ng-href="#/result/enter/{{ n.tournament_id }}/{{ n.player1_id }}/{{ n.player2_id }}" ng-show="yourId == n.player1_id || yourId == n.player2_id || n.is_league_manager == '1'">ENTER RESULT</a><br>
		</div>

		<div class="body">
			<span class="u-pull-right">{{ n.expected_percent }}%</span>
			<a ng-bind="n.player1_name" ng-href="#/profile/{{ n.player1_id }}"></a><span> ({{ n.player1_rating }})</span><br>
			<a ng-bind="n.player2_name" ng-href="#/profile/{{ n.player2_id }}"></a><span> ({{ n.player2_rating }})</span>
		</div>
	</div>

	<!-- Error message box if no fixtures -->
	<div class="error-container two-rem" ng-show="fixtures.length === 0">No fixtures yet</div>
</div>\end{lstlisting}
}
\textbf{home.html}

Path: /views/home.html
{\scriptsize
\begin{lstlisting}
<div class="container">

	<div class="box-container offset-by-two eight columns">
		<h1>A2 Computing Project</h1>
		<small>Jordan Lord</small>
	</div>

</div>\end{lstlisting}
}
\textbf{league-edit.html}

Path: /views/league-edit.html
{\scriptsize
\begin{lstlisting}
<div class="container">

	<div class="box-container seven columns" ng-class="{'offset-by-two eight columns': editOrCreate == 'Create'}">
		<h1>{{ editOrCreate }} League</h1>

		<!-- Error message box -->
		<div class="error-container" ng-hide="response.success">
			<ul>
				<li ng-bind="response.error_msg"></li>
			</ul>
		</div>

		<!-- League name field -->
		<label>League Name</label>
		<input class="u-full-width" type="text" ng-model="name">

		<!-- League description field -->
		<label>League Description</label>
		<textarea class="u-full-width" ng-model="description"></textarea>

		<!-- Delete league button -->
		<button class="u-full-width" ng-click="eventDelete()" ng-show="editOrCreate == 'Edit'">Delete League</button>

		<!-- Submit button -->
		<button class="u-full-width" ng-click="eventUpdate()">{{ editOrCreate }}</button>
	</div>

	<div class="box-container side five columns" ng-show="editOrCreate == 'Edit'">
		<!-- Display all league managers -->
		<div ng-hide="addingManager">
			<h2>Managers<a class="u-pull-right" href ng-click="eventAddManager()">Add</a></h2>

			<ul class="list-side">
				<li ng-repeat="n in data.league_managers">
					<a href="#/profile/{{ n.id }}" ng-bind="n.first_name + ' ' + n.last_name"></a>
					<a class="u-pull-right" href ng-click="eventRemoveManager(n.id)" ng-hide="n.id === yourId">Remove</a>
				</li>
			</ul>
		</div>

		<!-- Appears when adding a manager -->
		<div ng-show="addingManager">
			<h2>Add Manager<a class="u-pull-right" href ng-click="eventCancelAdding()">Cancel</a></h2>

			<!-- @directive cpUserSearch -->
			<div cp-user-search event-click-name="eventAddSpecificManager"></div>
		</div>

		<!-- Display all players -->
		<div ng-hide="addingPlayer">
			<h2>Players<a class="u-pull-right" href ng-click="eventAddPlayer()">Add</a></h2>

			<ul class="list-side">
				<li ng-repeat="n in data.players">
					<a href="#/profile/{{ n.id }}" ng-bind="n.first_name + ' ' + n.last_name"></a>
					<a class="u-pull-right" href ng-click="eventRemovePlayer(n.id)">Remove</a>
				</li>
			</ul>
		</div>

		<!-- Appears when adding a player -->
		<div ng-show="addingPlayer">
			<h2>Add Player<a class="u-pull-right" href ng-click="eventCancelAdding()">Cancel</a></h2>

			<!-- @directive cpUserSearch -->
			<div cp-user-search event-click-name="eventAddSpecificPlayer"></div>
		</div>
	</div>

</div>\end{lstlisting}
}
\textbf{league-search-page.html}

Path: /views/league-search-page.html
{\scriptsize
\begin{lstlisting}
<!-- Page where a user can search for a league -->
<div class="container">

	<div class="box-container offset-by-two eight columns">
		<h1>League Search</h1>

		<div class="create-league">
			<a href="#/league/create">Create a League</a>
		</div>

		<!-- Using @directive cpLeagueSearch to search for a league -->
		<div cp-league-search event-click-name="eventClickLeague"></div>
	</div>

</div>\end{lstlisting}
}
\textbf{league.html}

Path: /views/league.html
{\scriptsize
\begin{lstlisting}
<div class="container">

	<div class="box-container seven columns">
		<!-- Name of the league -->
		<h1 ng-bind="data.name">Premier League</h1>

		<!-- Edit button -->
		<a class="u-pull-right" href="#/league/{{ data.id }}/edit" ng-show="data.is_league_manager">Edit</a>

		<!-- Description of the league -->
		<p ng-bind="data.description"></p>


		<!-- Mini navigation -->
		<ul class="nav sub">
			<li><a href ng-click="subPage = 'table'" ng-class="{ active: subPage === 'table' }">Table</a></li>
			<li><a href ng-click="subPage = 'results'" ng-class="{ active: subPage === 'results' }">Results</a></li>
			<li><a href ng-click="subPage = 'fixtures'" ng-class="{ active: subPage === 'fixtures' }">Fixtures</a></li>
		</ul>


		<!-- Sub pages -->

		<!-- League table = subpage -->
		<div ng-show="subPage === 'table'">
			<table class="u-full-width league-table">
				<thead>
					<tr>
						<th width="10%">Pos</th>
						<th width="30%">Name</th>
						<th width="10%">P</th>
						<th width="10%">W</th>
						<th width="10%">L</th>
						<th>Rating</th>
						<th width="10%">Pts</th>
					</tr>
				</thead>
				<tbody>
					<tr ng-repeat="n in table track by $index">
						<td ng-bind="$index + 1">1</td>
						<td><a ng-href="#/profile/{{ n.user_id }}" ng-bind="n.name"></a></td>
						<td ng-bind="n.played"></td>
						<td ng-bind="n.wins"></td>
						<td ng-bind="n.loses"></td>
						<td ng-bind="n.rating"></td>
						<td ng-bind="n.points"></td>
					</tr>
				</tbody>
			</table>
		</div>


		<!-- Results = subpage -->
		<div cp-result tournament-id="data.id" is-ready="data.success" ng-show="subPage === 'results'"></div>

		
		<!-- Fixutes = subpage -->
		<div cp-fixtures tournament-id="data.id" is-ready="data.success" ng-show="subPage === 'fixtures'"></div>
	</div>


	<!-- Side bar -->
	<div class="five columns">
		<div class="box-container side">
			<!-- Display all league managers -->
			<h2>Managers</h2>

			<ul class="list-side">
				<li ng-repeat="n in data.league_managers">
					<a href="#/profile/{{ n.id }}" ng-bind="n.first_name + ' ' + n.last_name"></a>
				</li>
			</ul>
			
			<!-- Display all players -->
			<h2>Players</h2>

			<ul class="list-side">
				<li ng-repeat="n in data.players">
					<a href="#/profile/{{ n.id }}" ng-bind="n.first_name + ' ' + n.last_name"></a>
				</li>
			</ul>
		</div>
	</div>

</div>\end{lstlisting}
}
\textbf{login.html}

Path: /views/login.html
{\scriptsize
\begin{lstlisting}
<div class="container">

	<!-- Login form -->
	<form>
		<div class="row box-container six columns">
			<h1>Login</h1>

			<div class="error-container" ng-hide="response.success">
				<ul>
					<li ng-bind="response.error_msg"></li>
				</ul>
			</div>

			<!-- Email address field -->
			<div class="u-full-width">
				<label>Email Address</label>
				<input class="u-full-width" type="email" ng-model="inputEmailLogin">
			</div>

			<!-- Password field -->
			<div class="u-full-width">
				<label>Password</label>
				<input class="u-full-width" type="password" ng-model="inputPasswordLogin">
			</div>

			<!-- Login button -->
			<!-- When clicked, calls @method eventLogin -->
			<button class="u-full-width" ng-click="eventLogin()">Login</button>

		</div>
	</form>

	<!-- Register form -->
	<form>
		<div class="row box-container six columns u-pull-right">
			<h1>Register</h1>

			<div class="error-container" ng-hide="responseRegister.success">
				<ul>
					<li ng-bind="responseRegister.error_msg"></li>
				</ul>
			</div>

			<!-- First name field -->
			<div class="u-full-width">
				<label>First Name</label>
				<input class="u-full-width" type="text" ng-model="inputFirstName">
			</div>

			<!-- Last name field -->
			<div class="u-full-width">
				<label>Last Name</label>
				<input class="u-full-width" type="text" ng-model="inputLastName">
			</div>

			<!-- Email address field -->
			<div class="u-full-width">
				<label>Email Address</label>
				<input class="u-full-width" type="email" ng-model="inputEmailRegister">
			</div>

			<!-- Password field -->
			<div class="u-full-width">
				<label>Password</label>
				<input class="u-full-width" type="password" ng-model="inputPasswordRegister">
			</div>

			<!-- Repeat password field -->
			<div class="u-full-width">
				<label>Repeat Password</label>
				<input class="u-full-width" type="password" ng-model="inputPasswordRepeatRegister">
			</div>

			<!-- Register button -->
			<!-- When clicked, calls @method eventRegister -->
			<button class="u-full-width" ng-click="eventRegister()">Register</button>

		</div>
	</form>
</div>\end{lstlisting}
}
\textbf{profile-edit.html}

Path: /views/profile-edit.html
{\scriptsize
\begin{lstlisting}
<div class="container profile-edit">

	<div class="box-container offset-by-two eight columns">
		<h1>Edit Profile</h1>

		<!-- Where error messages go -->
		<div class="error-container" ng-hide="errorMsg === undefined">
			<ul>
				<li ng-bind="errorMsg"></li>
			</ul>
		</div>

		<!-- First name field -->
		<label>First Name</label>
		<input class="u-full-width" type="text" ng-model="first_name">

		<!-- Last name field -->
		<label>Last Name</label>
		<input class="u-full-width" type="text" ng-model="last_name">

		<!-- Home number field -->
		<!--
		<label>Home Phone Number</label>
		<input class="u-full-width" type="tel">
		-->

		<!-- Mobile number field -->
		<!--
		<label>Mobile Phone Number</label>
		<input class="u-full-width" type="tel">
		-->

		<!-- Submit button -->
		<button class="u-full-width" ng-click="eventSubmit()">Submit</button>
	</div>

</div>\end{lstlisting}
}
\textbf{profile.html}

Path: /views/profile.html
{\scriptsize
\begin{lstlisting}
<div class="container">

	<div class="box-container seven columns">
		<!-- Full name of user -->
		<h1 ng-bind="data.full_name"></h1>

		<!-- Edit button -->
		<a class="u-pull-right" href="#/profile/edit" ng-show="isUser">Edit</a>


		<!-- Mini navigation -->
		<ul class="nav sub">
			<li><a href ng-click="subPage = 'results'" ng-class="{active: subPage === 'results'}">Results</a></li>
			<li><a href ng-click="subPage = 'fixtures'" ng-class="{active: subPage === 'fixtures'}">Fixtures</a></li>
			<li><a href ng-click="subPage = 'progress'" ng-class="{active: subPage === 'progress'}">Progress</a></li>
		</ul>


		<!-- Sub pages -->

		<!-- Results = subpage -->
		<div ng-show="subPage === 'results'">
			<div cp-result player-one-id="data.id" is-ready="data.success"></div>
		</div>

		<!-- Fixutres = subpage -->
		<div ng-show="subPage === 'fixtures'">
			<div cp-fixtures player-id="data.id" is-ready="data.success"></div>
		</div>

		<!-- Progress = subpage -->
		<canvas id="canvas-rating-chart" class="u-full-width" width="500" height="500" ng-show="subPage === 'progress'"></canvas>
	</div>

	<!-- Side bar -->
	<div class="five columns">
		<div class="box-container side">
			<h2>Rating</h2>
			<span ng-bind="data.rating"></span>
			<div class="buffer"></div>
		</div>

		<div class="box-container side">
			<!-- Display all leagues that the user is playing in -->
			<h2>Leagues</h2>
			<ul class="list-side">
				<li ng-repeat="n in data.leagues_playing">
					<a href="#/league/{{ n.id }}" ng-bind="n.name"></a>
				</li>
			</ul>

			<!-- Display all leagues that the user is managing -->
			<h2>Managing</h2>
			<ul class="list-side">
				<li ng-repeat="n in data.leagues_managing">
					<a href="#/league/{{ n.id }}" ng-bind="n.name"></a>
				</li>
			</ul>

			<!--
			<h2>Achievements</h2>
			<span>Division 1 winner x2</span>
			-->
		</div>

		<!--
		<div class="box-container side">
			<h2>Contact</h2>
			<span>Home: 1234 4567</span><br>
			<span>Mobile: 07123456789</span>
		</div>
		-->
	</div>

</div>\end{lstlisting}
}
\textbf{result-enter.html}

Path: /views/result-enter.html
{\scriptsize
\begin{lstlisting}
<div class="container">

	<div class="box-container seven columns">
		<h1>Input Result</h1>

		<div class="error-container" ng-show="errorMsg.length != 0">
			<ul>
				<li ng-bind="errorMsg"></li>
			</ul>
		</div>

		<table>
			<tr>
				<td><h2 ng-bind="player1.fullName"></h2></td>
				<td><input type="number" ng-model="player1.score"></td>
			</tr>
			<tr>
				<td><h2 ng-bind="player2.fullName"></h2></td>
				<td><input type="number" ng-model="player2.score"></td>
			</tr>
		</table>

		<button class="u-full-width" ng-click="eventSubmitResult()">Submit</button>
	</div>

	<div class="box-container side five columns">
		<h2>League</h2>
		<a ng-href="#/league/{{ tournamentId }}" ng-bind="tournamentName"></a>
		<div class="buffer"></div>
		<h2>Date</h2>
		<span ng-bind="date"></span>
		<div class="buffer"></div>
	</div>

</div>\end{lstlisting}
}
\textbf{results.html}

Path: /views/results.html
{\scriptsize
\begin{lstlisting}
<!-- View for @directive cpResult -->
<div>
	<!-- Repeats for every result -->
	<div class="result" ng-repeat="n in results">
		<!-- Displays date and league -->
		<div class="head">
			<span ng-bind="n.date"></span>
			<a class="u-pull-right" href="#/league/{{ n.tournament_id }}" ng-bind="n.tournament_name"></a><br>
			<a class="u-pull-right" href ng-click="eventDelete(n.id)">Delete</a>
		</div>

		<!-- Displays player names and score -->
		<div class="body">
			<table class="u-full-width">
				<tr>
					<td width="40%">
						<a class="player1-name" href="#/profile/{{ n.player1_id }}" ng-bind="n.player1_name"></a>
					</td>
					<td width="20%" class="centre-align">
						<span class="score u-pull-center" ng-class="n.winOrLoss" ng-bind="n.score1 + '-' + n.score2"></span>
					</td>
					<td width="40%">
						<a class="player2-name" href="#/profile/{{ n.player2_id }}" ng-bind="n.player2_name"></a>
					</td>
				</tr>
				<tr>
					<td>
						<span class="player1-name">
							<span ng-bind="n.player1_rating"></span>
							<span>(<span ng-show="n.player1_rating_change > 0">+</span><span ng-bind="n.player1_rating_change"></span>)</span>
						</span>
					</td>
					<td>
					</td>
					<td>
						<span class="player2-name">
							<span ng-bind="n.player2_rating"></span>
							<span>(<span ng-show="n.player2_rating_change > 0">+</span><span ng-bind="n.player2_rating_change"></span>)</span>
						</span>
					</td>
				</tr>
			</table>
		</div>
	</div>

	<!-- Error message if there aren't any results to show -->
	<div class="error-container two-rem" ng-show="results.length === 0">No results yet</div>
</div>\end{lstlisting}
}
\textbf{search.html}

Path: /views/search.html
{\scriptsize
\begin{lstlisting}
<!-- View for the search directives -->
<div>
	<!-- Search field -->
	<input class="u-full-width" type="text" placeholder="Search" ng-model="inputName" ng-change="eventInputChange()"></input>

	<!-- Error shows when no league found -->
	<div class="error-container" ng-show="results.length === 0 && inputName.length !== 0">
		<ul>
			<li>No results</li>
		</ul>
	</div>

	<!-- Results from search -->
	<ul class="list-side" ng-show="inputName.length !== 0">
		<li ng-repeat="n in results">
			<a href ng-bind="n.name" ng-click="eventClickName(n.id)"></a>
		</li>
	</ul>
</div>\end{lstlisting}
}
\textbf{user-search-page.html}

Path: /views/user-search-page.html
{\scriptsize
\begin{lstlisting}
<!-- Page where a user can search for a user -->
<div class="container">

	<div class="box-container offset-by-two eight columns">
		<h1>User Search</h1>

		<!-- Using @directive cpUserSearch to search for a user -->
		<div cp-user-search event-click-name="eventClickUser"></div>
	</div>

</div>\end{lstlisting}
}
\newpage
\subsection{SQL}
\textbf{database.sql}

Path: /database.sql
{\scriptsize
\begin{lstlisting}

SET SQL_MODE = "NO_AUTO_VALUE_ON_ZERO";
SET time_zone = "+00:00";






CREATE TABLE IF NOT EXISTS `results` (
  `id` mediumint(8) unsigned NOT NULL AUTO_INCREMENT,
  `tournament_id` mediumint(8) unsigned NOT NULL,
  `date` datetime NOT NULL,
  PRIMARY KEY (`id`)
) ENGINE=InnoDB  DEFAULT CHARSET=latin1 AUTO_INCREMENT=1 ;



CREATE TABLE IF NOT EXISTS `result_user_maps` (
  `result_id` mediumint(8) unsigned NOT NULL,
  `user_id` mediumint(8) unsigned NOT NULL,
  `score` tinyint(3) unsigned NOT NULL,
  `rating` smallint(6) NOT NULL,
  `rating_change` smallint(6) NOT NULL,
  PRIMARY KEY (`result_id`, `user_id`)
) ENGINE=InnoDB DEFAULT CHARSET=latin1;



CREATE TABLE IF NOT EXISTS `tournaments` (
  `id` mediumint(8) unsigned NOT NULL AUTO_INCREMENT,
  `name` varchar(40) NOT NULL,
  `description` tinytext,
  PRIMARY KEY (`id`)
) ENGINE=InnoDB  DEFAULT CHARSET=latin1 AUTO_INCREMENT=1 ;



CREATE TABLE IF NOT EXISTS `tournament_user_maps` (
  `user_id` mediumint(8) unsigned NOT NULL,
  `tournament_id` mediumint(8) unsigned NOT NULL,
  `is_league_manager` tinyint(1) NOT NULL,
  `is_player` tinyint(1) NOT NULL,
  PRIMARY KEY (`user_id`,`tournament_id`)
) ENGINE=InnoDB DEFAULT CHARSET=latin1;



CREATE TABLE IF NOT EXISTS `users` (
  `id` mediumint(5) unsigned NOT NULL AUTO_INCREMENT,
  `email` varchar(255) NOT NULL UNIQUE,
  `first_name` varchar(30) NOT NULL,
  `last_name` varchar(30) NOT NULL,
  `hash` varchar(60) NOT NULL,
  `home_phone` varchar(20) DEFAULT NULL,
  `mobile_phone` varchar(20) DEFAULT NULL,
  PRIMARY KEY (`id`)
) ENGINE=InnoDB  DEFAULT CHARSET=latin1 AUTO_INCREMENT=1 ;

\end{lstlisting}
}
