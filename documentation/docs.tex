\subsection{PHP - Models}
\subsubsection{FixturesGet}
\textit{Model that returns fixture list for a tournament or user.
Lists all matches that are still to be played.}

Page \pageref{FixturesGet.php}

File /models/FixturesGet.php

Extends \texttt{Model}

\textbf{Properties:}

\texttt{query}

{\scriptsize
\textit{SQL query string that fetches the matches to be played.}

Type: String

Access: private

Line: 36

}
\textbf{Methods:}

\texttt{attachRatings()}

{\scriptsize
\textit{Method that gets the user ratings and puts them in every fixture including the expected result for player 1.}

Access: private

}

\texttt{main()}

{\scriptsize
\textit{Main method that execute @property query and binds input data.}

Access: protected

}

\subsubsection{Login}
\textit{Login model for logging into the system.}

Page \pageref{Login.php}

File /models/Login.php

Extends \texttt{Model}

\textbf{Properties:}

\texttt{query}

{\scriptsize
\textit{SQL query string for getting the details of the user.
Finds user by email.}

Type: String

Line: 21

}
\texttt{stmt}

{\scriptsize
\textit{Connects to the database.
Binds query and parameters.}

Type: Object

Access: private

Line: 34

}
\texttt{user}

{\scriptsize
\textit{Stores result from database.
User data.}

Type: Object

Access: private

Line: 43

}
\textbf{Methods:}

\texttt{storeSession()}

{\scriptsize
\textit{Stores the user details in the current user session.}

Access: private

}

\texttt{verify()}

{\scriptsize
\textit{Checks to see if password entered matches the password associated to the email entered.}

Access: private

}

\texttt{main()}

{\scriptsize
\textit{Defines @property result.
Binds parameters to @property query.}

Access: protected

}

\subsubsection{Logout}
\textit{Logout model called when the user logs out.}

Page \pageref{Logout.php}

File /models/Logout.php

Extends \texttt{Model}

\textbf{Methods:}

\texttt{main()}

{\scriptsize
\textit{Method that destroys the session.}

Access: protected

}

\subsubsection{Register}
\textit{Register model used to register onto the system.}

Page \pageref{Register.php}

File /models/Register.php

Extends \texttt{Model}

\textbf{Properties:}

\texttt{query}

{\scriptsize
\textit{SQL query string for inserting user data into the database.}

Type: String

Access: private

Line: 26

}
\texttt{query\_email}

{\scriptsize
\textit{SQL query string for checking whether email is already in use.}

Type: String

Access: private

Line: 38

}
\texttt{result}

{\scriptsize
\textit{For executing the query string.}

Type: Object

Access: private

Line: 51

}
\texttt{hash}

{\scriptsize
\textit{Hash of the password entered.}

Type: String

Access: private

Line: 59

}
\textbf{Methods:}

\texttt{bindParams()}

{\scriptsize
\textit{Used to bind the parameters to @property query.
Data inputted by user is bound.}

Access: private

}

\texttt{checkEmail()}

{\scriptsize
\textit{Method that will check whether email is already in use.}

}

\texttt{validate()}

{\scriptsize
\textit{Validate email, first name, last name, and password.}

Access: private

}

\texttt{main()}

{\scriptsize
\textit{Prepares query.
Hashes password.
Executes query.}

Access: public

}

\subsubsection{ResultEnter}
\textit{Model used for entering a result.}

Page \pageref{ResultEnter.php}

File /models/ResultEnter.php

Extends \texttt{Tournament}

\textbf{Properties:}

\texttt{query}

{\scriptsize
\textit{SQL query string for deleting a result.
Also deletes the users' association with the result.}

Type: String

Access: private

Line: 21

}
\texttt{result\_data}

{\scriptsize
\textit{Data associated with the result that the user wants to delete.}

Type: Array

Access: private

Line: 37

}
\texttt{query\_info}

{\scriptsize
\textit{SQL query string for inserting result info.}

Type: String

Access: private

Line: 27

}
\texttt{query\_score}

{\scriptsize
\textit{SQL query string for inserting result score.}

Type: String

Access: private

Line: 38

}
\texttt{id\_of\_result}

{\scriptsize
\textit{Id of the result that is being entered.}

Type: Integer

Access: private

Line: 50

}
\textbf{Methods:}

\texttt{ResultDelete()}

{\scriptsize
\textit{Model that is called to delete a particular result.}

}

\texttt{delete()}

{\scriptsize
\textit{Method that actually deletes the result.}

Access: private

}

\texttt{validate()}

{\scriptsize
\textit{Validates whether the user can delete the result.}

Access: private

}

\texttt{getResult()}

{\scriptsize
\textit{Method that gets the result data from @class ResultGet.}

Access: private

}

\texttt{main()}

{\scriptsize
\textit{Main method.}

Access: protected

}

\texttt{validScore()}

{\scriptsize
\textit{Method that checks whether the entered score is valid.}

Access: private

}

\texttt{validate()}

{\scriptsize
\textit{Main validation method.}

Access: private

}

\texttt{insertScore(user\_id, score, new\_rating, rating\_change)}

{\scriptsize
\textit{Method for inserting the score of a player.}

Access: private

}

\texttt{insertInfo()}

{\scriptsize
\textit{Method used to insert general result info.}

Access: private

}

\texttt{general()}

{\scriptsize
\textit{Calls all general methods.}

Access: private

}

\texttt{main()}

{\scriptsize
\textit{Checks login.}

Access: protected

}

\subsubsection{ResultGet}
\textit{Model for retrieving information on a particular result.}

Page \pageref{ResultGet.php}

File /models/ResultGet.php

Extends \texttt{Model}

\textbf{Properties:}

\texttt{query}

{\scriptsize
\textit{SQL query string for getting result information.}

Type: String

Access: private

Line: 37

}
\textbf{Methods:}

\texttt{main()}

{\scriptsize
\textit{Main method that executes @property query.}

Access: protected

}

\subsubsection{Status}
\textit{Model for getting the status of the current user.}

Page \pageref{Status.php}

File /models/Status.php

Extends \texttt{Model}

\textbf{Methods:}

\texttt{logged\_in()}

{\scriptsize
\textit{Method that stores user data in @property return\_data.}

Access: private

}

\texttt{main()}

{\scriptsize
\textit{Method that checks if the user is logged in.
Then sets @property return\_data.}

Access: protected

}

\subsubsection{TournamentCreate}
\textit{Model for users to create tournaments.}

Page \pageref{TournamentCreate.php}

File /models/TournamentCreate.php

Extends \texttt{Tournament}

\textbf{Properties:}

\texttt{query}

{\scriptsize
\textit{SQL query string for creating a tournament.}

Type: String

Access: private

Line: 23

}
\texttt{query\_add\_league\_manager}

{\scriptsize
\textit{Query for making the creator of the league a league manager.}

Type: String

Access: private

Line: 34

}
\texttt{stmt}

{\scriptsize
\textit{Statement object for executing @property query.}

Type: Object

Access: private

Line: 46

}
\texttt{tournament\_id}

{\scriptsize
\textit{Id of the tournament that has just been created.}

Type: 

Access: private

Line: 55

}
\textbf{Methods:}

\texttt{returnTournamentData()}

{\scriptsize
\textit{Method which returns the tournament data.
Data collected from @class TournamentGet.}

Access: private

}

\texttt{addLeagueManager()}

{\scriptsize
\textit{Method for attaching the current user as the league manager.}

Access: private

}

\texttt{create()}

{\scriptsize
\textit{Main function for creating database object, binding params, and executing query.}

Access: private

}

\texttt{main()}

{\scriptsize
\textit{Main function for checking whether user is logged in.}

Access: protected

}

\subsubsection{TournamentDelete}
\textit{Class that deletes a tournament.}

Page \pageref{TournamentDelete.php}

File /models/TournamentDelete.php

Extends \texttt{Tournament}

\textbf{Properties:}

\texttt{query\_league}

{\scriptsize
\textit{SQL query string for deleting the league.
Also deletes all associations with the league.
This includes results.}

Type: String

Access: private

Line: 19

}
\textbf{Methods:}

\texttt{delete()}

{\scriptsize
\textit{Method that uses @property query\_league to delete the league.}

Access: private

}

\texttt{verify()}

{\scriptsize
\textit{Checks whether the user can delete the league.}

Access: private

}

\texttt{main()}

{\scriptsize
\textit{Result of @method verify determines whether it calls @method delete.}

Access: protected

}

\subsubsection{TournamentGet}
\textit{Model that fetches tournament data based on id.}

Page \pageref{TournamentGet.php}

File /models/TournamentGet.php

Extends \texttt{Tournament}

\textbf{Properties:}

\texttt{query}

{\scriptsize
\textit{SQL query string for fetching tournament data.}

Type: String

Access: private

Line: 32

}
\texttt{query\_players}

{\scriptsize
\textit{SQL query string for fetching the players in the tournament.}

Type: String

Access: private

Line: 44

}
\texttt{query\_leauge\_managers}

{\scriptsize
\textit{SQL query string for fetching the league managers of the tournament.}

Type: String

Access: private

Line: 58

}
\textbf{Methods:}

\texttt{getPlayers()}

{\scriptsize
\textit{Method that queries players in the tournament.}

Access: private

}

\texttt{getLeagueManagers()}

{\scriptsize
\textit{Method that queries league managers in the tournament.}

Access: private

}

\texttt{getTournamentData()}

{\scriptsize
\textit{Method that gets the data about the tournament.}

Access: private

}

\texttt{main()}

{\scriptsize
\textit{Method that fetches the database info.}

Access: protected

}

\subsubsection{TournamentLeagueTable}
\textit{Class for generating a league table.}

Page \pageref{TournamentLeagueTable.php}

File /models/TournamentLeagueTable.php

Extends \texttt{Model}

\textbf{Properties:}

\texttt{table}

{\scriptsize
\textit{Array that stores the table data.}

Type: Array

Access: private

Line: 29

}
\texttt{query}

{\scriptsize
\textit{SQL query string for getting the players in the leauge.}

Type: String

Access: private

Line: 38

}
\textbf{Methods:}

\texttt{populateTable()}

{\scriptsize
\textit{If that player is unpopulated, then populate the table.}

Access: private

}

\texttt{calcuate()}

{\scriptsize
\textit{Calculate points and put in table.}

Access: private

}

\texttt{order()}

{\scriptsize
\textit{Order table and return.}

Access: private

}

\texttt{main()}

{\scriptsize
\textit{Method the executes @property query.
Returns league table.}

Access: protected

}

\subsubsection{TournamentManagerAttach}
\textit{Model for adding a user as a manager of a tournament.}

Page \pageref{TournamentManagerAttach.php}

File /models/TournamentManagerAttach.php

Extends \texttt{TournamentManagerAlter}

\textbf{Properties:}

\texttt{is\_league\_manager}

{\scriptsize
\textit{User becomes a league manager.}

Type: 

Access: protected

Line: 18

}
\textbf{Methods:}

\texttt{main()}

{\scriptsize
\textit{Main method calls @method subMain.}

Access: protected

}

\subsubsection{TournamentManagerRemove}
\textit{Model for removing a league manager from a tournament.}

Page \pageref{TournamentManagerRemove.php}

File /models/TournamentManagerRemove.php

Extends \texttt{TournamentManagerAlter}

\textbf{Properties:}

\texttt{is\_league\_manager}

{\scriptsize
\textit{No longer a league manager.}

Type: 

Access: protected

Line: 18

}
\textbf{Methods:}

\texttt{main()}

{\scriptsize
\textit{Main method that calls @method subMain.}

Access: protected

}

\subsubsection{TournamentPlayerAttach}
\textit{Model for adding a user as a player to a tournament.}

Page \pageref{TournamentPlayerAttach.php}

File /models/TournamentPlayerAttach.php

Extends \texttt{TournamentPlayerAlter}

\textbf{Properties:}

\texttt{is\_player}

{\scriptsize
\textit{Player becomes true when adding a player.}

Type: 

Access: protected

Line: 18

}
\textbf{Methods:}

\texttt{main()}

{\scriptsize
\textit{Calls @method subMain.}

Access: protected

}

\subsubsection{TournamentPlayerRemove}
\textit{Model for removing a player from a tournament.}

Page \pageref{TournamentPlayerRemove.php}

File /models/TournamentPlayerRemove.php

Extends \texttt{TournamentPlayerAlter}

\textbf{Properties:}

\texttt{is\_player}

{\scriptsize
\textit{Player becomes false when removing a player.}

Type: 

Access: protected

Line: 18

}
\textbf{Methods:}

\texttt{main()}

{\scriptsize
\textit{Calls @method subMain}

Access: protected

}

\subsubsection{TournamentSearch}
\textit{Model that is used to search for a tournament by name.}

Page \pageref{TournamentSearch.php}

File /models/TournamentSearch.php

Extends \texttt{Model}

\textbf{Properties:}

\texttt{query}

{\scriptsize
\textit{SQL query string for searching for tournaments and returning their name and id.}

Type: String

Access: private

Line: 22

}
\textbf{Methods:}

\texttt{main()}

{\scriptsize
\textit{Main method.
Used to execute @property query.}

Access: protected

}

\subsubsection{TournamentUpdate}
\textit{Model that updates tournament info.}

Page \pageref{TournamentUpdate.php}

File /models/TournamentUpdate.php

Extends \texttt{Tournament}

\textbf{Properties:}

\texttt{query}

{\scriptsize
\textit{SQL query string that updates tournament info.}

Type: String

Access: private

Line: 22

}
\textbf{Methods:}

\texttt{validate()}

{\scriptsize
\textit{Validates whether the user can update tournament info.}

Access: private

}

\texttt{update()}

{\scriptsize
\textit{Method that executes @property query.
Thus updating tournament info.}

Access: private

}

\texttt{main()}

{\scriptsize
\textit{Method that checks login, then calls @method update.}

Access: protected

}

\subsubsection{UserGet}
\textit{Model for querying a user based on the id.}

Page \pageref{UserGet.php}

File /models/UserGet.php

Extends \texttt{Model}

\textbf{Properties:}

\texttt{query}

{\scriptsize
\textit{SQL query string for fetching the user's data.}

Type: String

Access: private

Line: 32

}
\texttt{query\_managing}

{\scriptsize
\textit{SQL query string for getting tournaments that the user is managing.}

Type: String

Access: private

Line: 46

}
\texttt{query\_playing}

{\scriptsize
\textit{SQL query string for getting tournaments that the user is playing in.}

Type: String

Access: private

Line: 63

}
\texttt{stmt}

{\scriptsize
\textit{Database object for executing query.}

Type: Object

Access: private

Line: 80

}
\textbf{Methods:}

\texttt{verifyResult()}

{\scriptsize
\textit{Method that verifies whether the requested user is the one returned.}

Access: private

}

\texttt{executeQuery()}

{\scriptsize
\textit{Method that executes the query.}

Access: private

}

\texttt{getPlaying()}

{\scriptsize
\textit{Get tournament playing in.}

Access: private

}

\texttt{getManaging()}

{\scriptsize
\textit{Method for getting tournaments managing}

Access: private

}

\texttt{getUserData()}

{\scriptsize
\textit{Method for getting user data.}

Access: private

}

\texttt{main()}

{\scriptsize
\textit{Main method.
If the user exists, it gets tournament data.}

Access: protected

}

\subsubsection{UserRatings}
\textit{Model when called will return all the user's ratings over time.}

Page \pageref{UserRatings.php}

File /models/UserRatings.php

Extends \texttt{Model}

\textbf{Properties:}

\texttt{query}

{\scriptsize
\textit{SQL query string that fetches all the user's ratings over time.}

Type: String

Access: private

Line: 25

}
\textbf{Methods:}

\texttt{general()}

{\scriptsize
\textit{Method that executes @property query.}

Access: private

}

\texttt{main()}

{\scriptsize
\textit{Checks whether user is logged in then calls @method general.}

Access: protected

}

\subsubsection{UserSearch}
\textit{Model that is used to search for a user.}

Page \pageref{UserSearch.php}

File /models/UserSearch.php

Extends \texttt{Model}

\textbf{Properties:}

\texttt{query}

{\scriptsize
\textit{SQL query string for searching for users and returning their name and id.}

Type: String

Access: private

Line: 22

}
\textbf{Methods:}

\texttt{main()}

{\scriptsize
\textit{Main method.
Used to execute @property query.}

Access: protected

}

\subsubsection{UserUpdate}
\textit{This model is used to update user details.}

Page \pageref{UserUpdate.php}

File /models/UserUpdate.php

Extends \texttt{Model}

\textbf{Properties:}

\texttt{query}

{\scriptsize
\textit{SQL query string to update the user's data.}

Type: String

Access: private

Line: 24

}
\texttt{stmt}

{\scriptsize
\textit{Database object variable.}

Type: Object

Access: private

Line: 41

}
\textbf{Methods:}

\texttt{update()}

{\scriptsize
\textit{Main method once @method main has checked login.}

Access: private

}

\texttt{validate()}

{\scriptsize
\textit{Method that validates the inputs.
Validating first name, last name, and phone numbers.}

Access: private

}

\texttt{main()}

{\scriptsize
\textit{Checks to see if the user is logged in.
Calls @method update if true.}

Access: protected

}

\subsection{PHP - Important Classes}
\subsubsection{API}
\textit{Allows the JavaScript to communicate with the PHP models.}

Page \pageref{API.php}

File /php/API.php

\textbf{Properties:}

\texttt{modelName}

{\scriptsize
\textit{Name of the model to be called.}

Type: String

Access: private

Line: 10

}
\textbf{Methods:}

\texttt{execute()}

{\scriptsize
\textit{Creates an instance of the class requested.
Therefore calling the model.}

Access: private

}

\texttt{getData()}

{\scriptsize
\textit{Gets the data from GET or POST.}

Access: private

}

\texttt{requireAll()}

{\scriptsize
\textit{Method that requires all the files.}

Access: private

}

\texttt{\_\_construct()}

{\scriptsize
\textit{Method that is executed when the API is called.
Calls all of the other methods.}

Access: public

}

\subsubsection{Database}
\textit{The Database class connects the the database when the file is included.}

Page \pageref{Database.php}

File /php/Database.php

\textbf{Properties:}

\texttt{conn}

{\scriptsize
\textit{This variable is the connection variable for connecting to the database.}

Type: Object

Line: 10

}
\texttt{query\_delete}

{\scriptsize
\textit{SQL query string for deleting all tables}

Type: String

Access: private

Line: 18

}
\textbf{Methods:}

\texttt{create()}

{\scriptsize
\textit{This method is used to create the tables and columns in the database.
The `database.sql` file provides the SQL query string to do this.}

Access: public

}

\texttt{delete()}

{\scriptsize
\textit{Method that deletes all the tables in the database.}

Access: public

}

\texttt{reset()}

{\scriptsize
\textit{Resets database.
Deletes then recreates.}

Access: public

}

\texttt{init()}

{\scriptsize
\textit{This method is called when the file is included.
It is used to connect to the database.
Also to set @property conn to new PDO.}

Access: public

}

\subsubsection{Model}
\textit{Model class, the class is extended by all models.
It provides a foundation for all models.}

Page \pageref{Model.php}

File /php/Model.php

\textbf{Properties:}

\texttt{name}

{\scriptsize
\textit{The name of the data variable in POST.}

Type: 

Access: private

Line: 11

}
\texttt{data}

{\scriptsize
\textit{The data object is used to store.}

Type: Array

Access: protected

Line: 19

}
\texttt{return\_data}

{\scriptsize
\textit{The object for holding all data that wants to be returned.}

Type: Array

Access: protected

Line: 28

}
\texttt{success}

{\scriptsize
\textit{Whether model executed successfully.}

Type: {Boolean} Default true.

Access: protected

Line: 37

}
\texttt{error\_msg}

{\scriptsize
\textit{Error message string, if error.}

Type: String

Access: protected

Line: 46

}
\textbf{Methods:}

\texttt{returnObj()}

{\scriptsize
\textit{Method that assembles the return object.}

Access: private

}

\texttt{decodePost()}

{\scriptsize
\textit{This method is used to decode the JSON data in the post.}

Access: private

}

\texttt{isPost()}

{\scriptsize
\textit{This returns whether the post data variable is set.}

Access: private

}

\texttt{setVars()}

{\scriptsize
\textit{Resets variables when model is called.}

Access: private

}

\texttt{call(Data)}

{\scriptsize
\textit{Call allows PHP to pass data into the model.}

Access: public

}

\texttt{\_\_construct(notAPI)}

{\scriptsize
\textit{Function that is called to check if it is called with Post.}

Access: public

}

\subsubsection{Test}
\textit{Class for testing PHP models.}

Page \pageref{Test.php}

File /php/Test.php

\textbf{Methods:}

\texttt{requireAll()}

{\scriptsize
\textit{Method that requires all the files.}

Access: private

}

\texttt{testStart(testName)}

{\scriptsize
\textit{Method that should be called when starting a test.}

Access: private

}

\texttt{testEnd()}

{\scriptsize
\textit{Method that should be called when finished a test.}

Access: private

}

\texttt{unitTest()}

{\scriptsize
\textit{When making a single test, call this method.
Puts row in table.}

Access: private

}

\texttt{loadTest()}

{\scriptsize
\textit{Method that loads the test.}

Access: private

}

\texttt{init()}

{\scriptsize
\textit{Method that is called first.
Calls all tests individually.}

Access: public

}

\subsection{PHP - Parent Classes}
\subsubsection{EloRating}
\textit{Class with functions for calculating new elo rating.}

Page \pageref{EloRating.php}

File /superclasses/EloRating.php

\textbf{Properties:}

\texttt{new\_rating}

{\scriptsize
\textit{New rating of the player.}

Type: Integer

Access: public

Line: 13

}
\texttt{rating\_change}

{\scriptsize
\textit{Rating change.
New rating - old rating.}

Type: Integer

Access: public

Line: 21

}
\texttt{k\_factor}

{\scriptsize
\textit{K factor for the weight of rating change.}

Type: Integer

Access: public

Line: 30

}
\texttt{query\_get\_rating}

{\scriptsize
\textit{SQL query for getting the player's latest rating.}

Type: String

Access: private

Line: 40

}
\texttt{default\_rating}

{\scriptsize
\textit{Start rating for all players.}

Type: Integer

Access: public

Line: 60

}
\textbf{Methods:}

\texttt{getPlayerRating(user\_id)}

{\scriptsize
\textit{Method gets rating of a user using @property query\_get\_rating.}

Access: public

}

\texttt{expected(rating\_a, rating\_b)}

{\scriptsize
\textit{Method that calculates a probability of a player winning.}

Access: public

}

\texttt{newRating(rating\_a, k\_factor, points\_a, expected\_a)}

{\scriptsize
\textit{Method that calculates a new rating based on the score.}

Access: public

}

\texttt{\_\_construct(player\_a\_id, player\_b\_id, tournament\_id, player\_a\_score, player\_b\_score)}

{\scriptsize
\textit{Method that is called when an instance of the class is made.}

Access: public

}

\subsubsection{ResultMethods}
\textit{Helpful functions for dealing with results.}

Page \pageref{ResultMethods.php}

File /superclasses/ResultMethods.php

\textbf{Properties:}

\texttt{query\_result\_exists}

{\scriptsize
\textit{SQL query that checks whether a result exists or not.}

Type: String

Access: private

Line: 13

}
\textbf{Methods:}

\texttt{resultExists(player1\_id, player2\_id, tournament\_id)}

{\scriptsize
\textit{Method that tests whether a match has already been played.}

Access: public

}

\subsubsection{Tournament}
\textit{Parent class for general tournament stuff.}

Page \pageref{Tournament.php}

File /superclasses/Tournament.php

Extends \texttt{Model}

\textbf{Properties:}

\texttt{query\_players}

{\scriptsize
\textit{SQL query string for fetching the players in the tournament.}

Type: String

Access: private

Line: 15

}
\texttt{query\_player\_count}

{\scriptsize
\textit{SQL query string for checking whether the user is a player in a particular league.}

Type: String

Access: private

Line: 29

}
\texttt{query\_user\_exist}

{\scriptsize
\textit{SQL query string for telling whether a user exists.}

Type: String

Access: private

Line: 44

}
\texttt{query\_league\_manager\_count}

{\scriptsize
\textit{SQL query string for checking whether the user is league manager of a particular league.}

Type: String

Access: private

Line: 56

}
\texttt{query\_tournament\_count}

{\scriptsize
\textit{SQL query string for checking whether a tournament exists.}

Type: String

Access: private

Line: 71

}
\texttt{query\_attach}

{\scriptsize
\textit{SQL query for attaching a user to a tournament.}

Type: String

Access: private

Line: 83

}
\textbf{Methods:}

\texttt{isPlayer(user\_id, tournament\_id)}

{\scriptsize
\textit{Method for telling whether the current user is a player in the tournament.}

Access: protected

}

\texttt{isLeagueManager(user\_id, tournament\_id)}

{\scriptsize
\textit{Method for telling whether the current user is a league manager is the tournament.}

Access: protected

}

\texttt{attachUser(tournament\_id, user\_id)}

{\scriptsize
\textit{Attaches a user to a tournament.}

Access: protected

}

\texttt{tournamentExistsId(tournament\_id)}

{\scriptsize
\textit{Checks whether a tournament exists from parameter id.}

Access: protected

}

\texttt{tournamentExists()}

{\scriptsize
\textit{Checks whether a tournament exists.}

Access: protected

}

\texttt{userExists(user\_id)}

{\scriptsize
\textit{Checks whether a user exists.}

Access: protected

}

\subsubsection{TournamentManagerAlter}
\textit{Parent class of classes that add or remove league managers.}

Page \pageref{TournamentManagerAlter.php}

File /superclasses/TournamentManagerAlter.php

Extends \texttt{Tournament}

\textbf{Properties:}

\texttt{query}

{\scriptsize
\textit{SQL query string for changing league manager status.}

Type: String

Access: private

Line: 17

}
\texttt{query\_managers\_count}

{\scriptsize
\textit{SQL query string for counting the number of league managers.}

Type: String

Access: private

Line: 32

}
\texttt{stmt}

{\scriptsize
\textit{Database object for executing @property query.}

Type: Object

Access: private

Line: 47

}
\textbf{Methods:}

\texttt{noOfLeagueManagers()}

{\scriptsize
\textit{Method for finding the number of league manager.}

Access: private

}

\texttt{executeQuery()}

{\scriptsize
\textit{Checks any faults and executes query.}

Access: private

}

\texttt{verifyManager()}

{\scriptsize
\textit{Verifies whether the user can carry out the task.
Returns false if:
- User doesn't exist.
- Tournament doesn't exist.
- Not a league manager.}

Access: private

}

\texttt{general()}

{\scriptsize
\textit{Main method for making the user a league manager}

Access: private

}

\texttt{subMain()}

{\scriptsize
\textit{Method that checks whether the user is logged in.}

Access: protected

}

\subsubsection{TournamentPlayerAlter}
\textit{Parent class of classes that add or remove players.}

Page \pageref{TournamentPlayerAlter.php}

File /superclasses/TournamentPlayerAlter.php

Extends \texttt{Tournament}

\textbf{Properties:}

\texttt{query}

{\scriptsize
\textit{SQL query string for updating a player}

Type: String

Access: private

Line: 16

}
\texttt{stmt}

{\scriptsize
\textit{Database object for executing @property query.}

Type: Object

Access: private

Line: 31

}
\textbf{Methods:}

\texttt{executeQuery()}

{\scriptsize
\textit{Checks any faults and executes query.}

Access: private

}

\texttt{verifyPlayer()}

{\scriptsize
\textit{Method for verifying whether the user can carry out the task.
Returns false if:
- User doesn't exist.
- Tournament doesn't exist.
- Either:
  - Not a league manager.
  - Altering someone else.}

Access: private

}

\texttt{general()}

{\scriptsize
\textit{Method that creates database object.
Binds parameters.
Checks to see if able to execute query.
Then execute query.}

Access: private

}

\texttt{subMain()}

{\scriptsize
\textit{Method that checks login.
Then calls @method general}

Access: protected

}

\subsubsection{Validate}
\textit{Class used to validate input text.}

Page \pageref{Validate.php}

File /superclasses/Validate.php

\textbf{Methods:}

\texttt{returnData(success, error\_msg)}

{\scriptsize
\textit{Returns standard data such as whether success, error message, and correction.}

Access: private

}

\texttt{userName(name, variableName)}

{\scriptsize
\textit{Valididates user's name.
Can be first or last name.}

}

\texttt{tournamentName(name)}

{\scriptsize
\textit{Validates tournament name.}

}

\texttt{tournamentDescription(description)}

{\scriptsize
\textit{Validates tournament description.}

}

\texttt{email(email)}

{\scriptsize
\textit{Validates email address.}

}

\texttt{password(password)}

{\scriptsize
\textit{Validates password.}

}

\texttt{phoneNumber(phoneNumber)}

{\scriptsize
\textit{Validates phone number.}

}

\subsection{JavaScript - Services}
\subsubsection{CallModel}
\textit{Factory for calling the PHP models.}

Page \pageref{CallModel.js}

File /services/CallModel.js

\textbf{Methods:}

\texttt{fetch(modelName, data, callbacks, success, fail, normal)}

{\scriptsize
\textit{Method for fetching a result from a model.
Sends data to the model.}

}

\texttt{ifLoggedIn(callbackTrue, callbackFalse)}

{\scriptsize
\textit{Method that accepts two functions.
One is executed if the user is logged in.
The other is called if the user is not logged in.}

}

\texttt{redirectIfNotLoggedIn()}

{\scriptsize
\textit{Method that redirects to login page if not logged in.}

}

\subsubsection{DateFormat}
\textit{Factory that returns the site's format of date.}

Page \pageref{DateFormat.js}

File /services/DateFormat.js

\textbf{Properties:}

\texttt{days}

{\scriptsize
\textit{Days of the week for date format.}

Type: Array

Line: 8

}
\texttt{months}

{\scriptsize
\textit{Months of the year for date format.}

Type: Array

Line: 15

}
\textbf{Methods:}

\texttt{getDateString(dateString)}

{\scriptsize
\textit{Function that returns formated output date from input date.}

}

\subsubsection{RatingChart}
\textit{Factory for handling the rating graph.}

Page \pageref{RatingChart.js}

File /services/RatingChart.js

\textbf{Properties:}

\texttt{ctx}

{\scriptsize
\textit{}

Type: Object

Line: 8

}
\texttt{chart}

{\scriptsize
\textit{Chart object for talking to Chart.JS library.}

Type: Object

Line: 13

}
\texttt{dates}

{\scriptsize
\textit{List of dates corresponding to a rating.}

Type: Array

Line: 20

}
\texttt{averageRating}

{\scriptsize
\textit{List of ratings for the graph.}

Type: Array

Line: 27

}
\texttt{data}

{\scriptsize
\textit{Data object for drawing the chart.}

Type: Object

Line: 34

}
\texttt{options}

{\scriptsize
\textit{Configuration for drawing the chart.}

Type: Object

Line: 55

}
\textbf{Methods:}

\texttt{inputRatings(ratings)}

{\scriptsize
\textit{Method that filters the ratings into different tournaments.}

}

\texttt{init()}

{\scriptsize
\textit{Call this method when initialising the graph.}

}

\texttt{draw()}

{\scriptsize
\textit{This method is called when drawing the graph.}

}

\subsection{JavaScript - Directives}
\subsubsection{cpFixtures}
\textit{Directive for displaying fixtures.}

Page \pageref{cpFixtures.js}

File /directives/cpFixtures.js

\textbf{Properties:}

\texttt{self}

{\scriptsize
\textit{Variable for storing scope.}

Type: Object

Line: 8

}
\textbf{Methods:}

\texttt{getFixtures()}

{\scriptsize
\textit{Method that gets data about fixtures from @class FixturesGet.}

}

\subsubsection{cpLeagueSearch}
\textit{Directive used to search for a tournament.}

Page \pageref{cpLeagueSearch.js}

File /directives/cpLeagueSearch.js

\textbf{Properties:}

\texttt{scope.inputName}

{\scriptsize
\textit{Must be empty string for search results box to fully hide}

Type: String

Line: 15

}
\textbf{Methods:}

\texttt{scope.eventInputChange()}

{\scriptsize
\textit{Method that is called when text in input field is changed.
Updates search for tournament when input is changed.}

}

\subsubsection{cpResult}
\textit{Directive for handling results.}

Page \pageref{cpResult.js}

File /directives/cpResult.js

\textbf{Properties:}

\texttt{self}

{\scriptsize
\textit{Variable for storing scope.}

Type: 

Line: 8

}
\textbf{Methods:}

\texttt{sortResults()}

{\scriptsize
\textit{Function that sets CSS class for colouring result.
Also calls @method getDateString for each result.}

}

\texttt{getResults()}

{\scriptsize
\textit{Function that gets the results data from @class ResultGet.}

}

\texttt{\$scope.eventDelete(id)}

{\scriptsize
\textit{Function that deletes a result when delete button is clicked.}

}

\subsubsection{cpUserSearch}
\textit{Directive used to search for a user.}

Page \pageref{cpUserSearch.js}

File /directives/cpUserSearch.js

\textbf{Properties:}

\texttt{scope.inputName}

{\scriptsize
\textit{Search name}

Type: String

Line: 15

}
\textbf{Methods:}

\texttt{scope.eventInputChange()}

{\scriptsize
\textit{Method that is called when text in input field is changed.
Updates search for user when input is changed.}

}

\subsection{JavaScript - Controllers}
\subsubsection{HomeCtrl}
\textit{Controller for the home page.}

Page \pageref{HomeCtrl.js}

File /controllers/HomeCtrl.js

\subsubsection{LeagueCreateCtrl}
\textit{Controller for creating a league.}

Page \pageref{LeagueCreateCtrl.js}

File /controllers/LeagueCreateCtrl.js

\textbf{Properties:}

\texttt{\$scope.editOrCreate}

{\scriptsize
\textit{The text that goes in the edit page.}

Type: String

Line: 8

}
\texttt{\$scope.response}

{\scriptsize
\textit{Object where error message is and tells whether there is an error message.}

Type: Object

Line: 15

}
\textbf{Methods:}

\texttt{\$scope.eventUpdate()}

{\scriptsize
\textit{Event that is called when 'Create' button is clicked.
Creating the league.}

}

\subsubsection{LeagueCtrl}
\textit{League page controller.}

Page \pageref{LeagueCtrl.js}

File /controllers/LeagueCtrl.js

\textbf{Properties:}

\texttt{\$scope.subPage}

{\scriptsize
\textit{String that determines which sub page the user is on.}

Type: String

Line: 8

}
\texttt{\$scope.table}

{\scriptsize
\textit{Object that contains the table data.}

Type: Object

Line: 15

}
\textbf{Methods:}

\texttt{getLeague()}

{\scriptsize
\textit{Method for getting league data from @class TournamentGet.}

}

\subsubsection{LeagueEditCtrl}
\textit{Controller for editing a league.}

Page \pageref{LeagueEditCtrl.js}

File /controllers/LeagueEditCtrl.js

\textbf{Properties:}

\texttt{\$scope.editOrCreate}

{\scriptsize
\textit{The text that goes in the edit page.}

Type: String

Line: 8

}
\texttt{\$scope.yourId}

{\scriptsize
\textit{Id of the user.
Used to prevent the user from removing themselves as a league manager.}

Type: Integer

Line: 15

}
\texttt{\$scope.addingManager}

{\scriptsize
\textit{Variable that is used to tell the view whether the adding manager dialog is open.}

Type: Boolean

Line: 23

}
\texttt{\$scope.addingPlayer}

{\scriptsize
\textit{Variable that is used to tell the view whether the adding player dialog is open.}

Type: Boolean

Line: 30

}
\texttt{\$scope.response}

{\scriptsize
\textit{Object where error message is and tells whether there is an error message.}

Type: Object

Line: 37

}
\textbf{Methods:}

\texttt{redirect()}

{\scriptsize
\textit{Method that redirects to home page and prints error message.}

}

\texttt{getLeague()}

{\scriptsize
\textit{Method for getting league data from @class TournamentGet.}

}

\texttt{\$scope.eventUpdate()}

{\scriptsize
\textit{Function that is called when the update button is pressed.}

}

\texttt{\$scope.eventRemoveUser(userId)}

{\scriptsize
\textit{Function that is called when a player is removed from the league.}

}

\texttt{\$scope.eventRemoveManager(userId)}

{\scriptsize
\textit{Function that is called when a league manager is removed.}

}

\texttt{\$scope.eventAddSpecificPlayer(userId)}

{\scriptsize
\textit{Function that is called by @directive cpUserSearch when adding a player.}

}

\texttt{\$scope.eventAddSpecificManager(userId)}

{\scriptsize
\textit{Function that is called by @directive cpUserSearch when adding a league manager.}

}

\texttt{\$scope.eventAddManager()}

{\scriptsize
\textit{Function that opens adding league manager dialog but closes others.}

}

\texttt{\$scope.eventAddPlayer()}

{\scriptsize
\textit{Function that opens adding player dialog but closes others.}

}

\texttt{\$scope.eventCancelAdding()}

{\scriptsize
\textit{Function that closes all adding user dialogs.}

}

\texttt{\$scope.eventDelete()}

{\scriptsize
\textit{Function that calls @class TournamentDelete, to delete the league.}

}

\subsubsection{UserSearchCtrl}
\textit{Controller used to search for a user.}

Page \pageref{UserSearchCtrl.js}

File /controllers/UserSearchCtrl.js

\textbf{Methods:}

\texttt{\$scope.eventClickLeague(leagueId)}

{\scriptsize
\textit{Method that is called when a league is clicked.
Clicked from @directive cpLeagueSearch.}

}

\texttt{\$scope.eventClickUser(userId)}

{\scriptsize
\textit{Method that is called when a user is clicked.
Clicked from @directive cpUserSearch.}

}

\subsubsection{LoginCtrl}
\textit{Controller for managing user login.
Also manages user registering.}

Page \pageref{LoginCtrl.js}

File /controllers/LoginCtrl.js

\textbf{Properties:}

\texttt{\$scope.response}

{\scriptsize
\textit{States the success of response.}

Type: Object

Line: 9

}
\texttt{\$scope.responseRegister}

{\scriptsize
\textit{States the success of the response from registering.}

Type: Object

Line: 19

}
\textbf{Methods:}

\texttt{getUserData()}

{\scriptsize
\textit{Method for getting user data and storing it in a session.}

}

\texttt{eventLogin()}

{\scriptsize
\textit{Method that logs the user in.}

}

\texttt{eventRegister()}

{\scriptsize
\textit{Method that registers the user.}

}

\subsubsection{NavCtrl}
\textit{Controller for managing the navbar.}

Page \pageref{NavCtrl.js}

File /controllers/NavCtrl.js

\textbf{Methods:}

\texttt{loginButtonText()}

{\scriptsize
\textit{Changes login boolean depending whether logged in or not.}

}

\texttt{\$scope.btnLogout()}

{\scriptsize
\textit{Event when logout button is clicked.
Calls model 'Logout' then redirects to login page.}

}

\subsubsection{ProfileCtrl}
\textit{Profile page controller.}

Page \pageref{ProfileCtrl.js}

File /controllers/ProfileCtrl.js

\textbf{Properties:}

\texttt{userId}

{\scriptsize
\textit{Id of the user.}

Type: Integer

Line: 8

}
\texttt{\$scope.currentSubPage}

{\scriptsize
\textit{Variable for storing the name of the current subpage.}

Type: String

Line: 16

}
\texttt{\$scope.isUser}

{\scriptsize
\textit{Is user.}

Type: Boolean

Line: 23

}
\textbf{Methods:}

\texttt{getUser(user\_id)}

{\scriptsize
\textit{Method that gets user data from UserGet model.}

}

\texttt{getStatus(callback)}

{\scriptsize
\textit{Method the gets the id of the current user.
Then calls @method getUser with id of current user.}

}

\texttt{getRatings()}

{\scriptsize
\textit{Method that gets the users ratings for the graph.}

}

\subsubsection{ProfileEditCtrl}
\textit{Controller for editing a user.}

Page \pageref{ProfileEditCtrl.js}

File /controllers/ProfileEditCtrl.js

\textbf{Properties:}

\texttt{\$scope.yourId}

{\scriptsize
\textit{The user id.}

Type: Integer

Line: 8

}
\texttt{\$scope.errorMsg}

{\scriptsize
\textit{Error message to be displayed if there is an error.}

Type: String

Line: 15

}
\textbf{Methods:}

\texttt{\$scope.eventSubmit()}

{\scriptsize
\textit{Method that is called when submit button is pressed.}

}

\texttt{getUserData()}

{\scriptsize
\textit{Method the gets the data of the user.}

}

\subsubsection{ResultEnterCtrl}
\textit{Controller that allows a user to input a result.}

Page \pageref{ResultEnterCtrl.js}

File /controllers/ResultEnterCtrl.js

\textbf{Properties:}

\texttt{\$scope.tournamentId}

{\scriptsize
\textit{The id of the tournament as defined by the route parametre.}

Type: Integer

Line: 8

}
\texttt{\$scope.tournamentName}

{\scriptsize
\textit{Name of the tournament.}

Type: String

Line: 15

}
\texttt{\$scope.player1}

{\scriptsize
\textit{Player 1 data object.}

Type: Integer

Line: 22

}
\texttt{\$scope.player2}

{\scriptsize
\textit{Player 2 data object.}

Type: Integer

Line: 29

}
\texttt{\$scope.date}

{\scriptsize
\textit{The date that the result will be entered.}

Type: String

Line: 36

}
\texttt{\$scope.errorMsg}

{\scriptsize
\textit{Error message, if there is one.}

Type: String

Line: 43

}
\textbf{Methods:}

\texttt{\$scope.eventSubmitResult()}

{\scriptsize
\textit{Method that is called when submit button is pressed.
Enters a result.}

}

\texttt{getTournamentData()}

{\scriptsize
\textit{Method for getting information about the tournament.}

}

\texttt{getPlayerData()}

{\scriptsize
\textit{Method that gets the player data.}

}

